\chapter{Lähtökohdat ja tarve}
\label{ch:lähtökohdat ja tarve}
Kirjoita tähän siitä miksi tällainen toteutus tarvitaan ja miksi. Pohjusta miksi suunniteltava ohjelmisto tarvitaan toteuttaa yritykseen johon työn teen. Ohjelmiston tarkoitus olisi tilata IEC 61850 -standardin määrittämiä raportteja ja muokata ne uuteen muotoon (JSON) ja julkaista ne eteenpäin jonoon toiselle tilaavalle ohjelmalle käyttäen AMQP-standardin määrittämää viestintää. Jonon tilaava asiakasohjelmisto voi olla mikä tahansa muu ohjelmisto.

Kirjoita tätä ensin ja listaa eksplisiittisesti mitä ollaan tekemässä ja asioita. Voi aloita heti kirjoittaa ennen kun otsikon päättää. Aseta työlle isoja kysymyksiä joidin etsitään vastausta. Kirjoita tämä ennen seuraava palaveriä. Lisäksi ota huomioon muut dokkarissa olevat kommentit proffalta.

Muistiinpanoja palaverista:
\begin{quote}
Mieti raportti sanaa vaihtaa johonkin muuhun kun kyseessä on kuitenkin koneelta koneelle jutelua. Raporti-sanasta tulee enemmän mieleen projektiraportti lukijalle. Lopussa peilaa tuloksia asetettuihin kymyksiin. Hyvälle dipalle pitäisi asettaa "ympyrä". Eli ensin asetetaan alkutilanne ja isot kysymkset aiheeseen liittyen. Sitten mennään tutkimuksen ja aiheen kautta tuloksiin. Lopussa tuloksia peilataan alussa asetettuihin kysymksiin.

Juhannus aikaan pari viikkoa poissa. Heinäkuun loppupuolella paremmin tavoitettavissa. Laittaa varmaan spostia milloin on tavoitettavissa.
\end{quote}


