\chapter{Yhteenveto}
\label{ch:yhteenveto}
Tässä diplomityössä lähdettiin etsimään ratkaisua kuinka suunnitellaan ja toteutetaan yksittäinen komponentti osaksi isompaa järjestelmää. Lähtökohtana oli, että komponentti pystyisi tilaamaan tietoa sähköseman IED-laitteelta IEC 61850 -standardin mukaisesti. Lisäksi tieto täytyi saada jaettua järkevästi järjestelmän muille komponenteille, ilman että päädyttäisiin huonoihin teknisiin ratkaisuihin. Ennen diplomityön aloitusta tekijä oli yrityksessä toteuttanut demon komponentin toimivuudesta. Demo oli samalla tie oppia monimutkaisen IEC 61850 -standardin toiminta ja perehtyä aiheeseen tarkemmin ennen oikeaa toteutusta. Tämä diplomityö keskittyi komponentin uuden version suunnitteluun ja toteutukseen käyttäen hyväksi demototeutuksen analyysia toiminnallisuudesta ja ongelmista. Lisäksi työn alussa asetettiin tutkimuskysymyksiä aiheeseen liittyen, joihin pyrittiin etsimään vastausta. Tutkimuskysymykset esiteltiin työn alussa kappalessa \ref{ch:johdanto}.

Työssä etsittiin ratkaisua siihen, mikät olisivat sopivat ohjelmiston arkkitehtuurimallit tämän kaltaisen ongelman ratkaisuun. Toteutuksessa päädyttiin käyttämään tilaaja-julkaisija arkkitehtuurimallia, missä toteutettu komponentti on julkaisija ja muut järjestelmän komponentit ovat tilaajia. Sähköaseman IED-laitteen ja toteutuksen välinen tiedonjako täytyy automaattisesti noudattaa tilaaja-julkaisija arkkitehtuuria, koska IEC 61850 -standardissa määritetään niin. Tähän toteutusmalliin päädyttiin, koska järjestelmän muut komponentit tarvitsevat tietoa samalla periaatteella kuin ulkopuolinen ohjelma IED-laitteelta. Ja tilaaja-julkaisija malli jatkaa standardissa määritettyn mallin toimivuutta, mutta jakaa saman tiedon monen osapuolen kanssa. Lisäksi arkkitehtuurimalli on tarkoitettu ratkomaan tämän kaltaisia tilanteita. Näin saatiin aikaiseksi tiedon kulku järjestelmässä yhteen suuntaan ilman kyseenalaisia teknisiä toteutuksia.

IED-laitteelta tuleva viesti oli tarkoitus saada jaettua järjestelmän muiden komponenttien kanssa. Demossa tämä oli toteutettu käyttäen tietokantaa tiedon välittäjänä, mikä ei ollut hyvä ratkaisu. Työn toteutuksessa viestien välitykseen valittiin RabbitMQ-välittäjäpalvelin, joka myös toimii tilaaja-julkaisija arkkitehtuurin mukaisesti. Tämän avulla järjestelmän muut komponentit voivat tilata viestejä tarpeidensa mukaan ja saavat siitä huomautuksen kun uusi viesti saapuu. Lisäksi palvelin tarjoaa sisäisesti jonot tilaajille, jos ne eivät kerkeä prosessoida viestejä tarpeeksi nopeasti. RabbitMQ viestien välitykseen järjestelmän muille komponteille osoittautui järkeväksi ja toimivaksi vaihtoehdoksi.

Viestin muoto IED-laitteelta noudattaa IEC 61850 -standardin määrittämää muotoa. Toteutuksessa tämän viestin lukeminen hoidettiin libiec61850-kirjastolla. Jotta tieto saataisiin järjestelmän muille komponenteille järkevästi, täytyi se muuttaa johonkin muuhun helppokäyttöisempään muotoon. Tämä myös sen takia jos järjestelmän eri osia toteutetaan eri tekniikoilla, olisi viesti helppo lukea tekniikasta riippumatta. Toteutuksessa viesti päädyttiin muuttamaan JSON-muotoon. JSON on nykypäivänä paljon käytetty tiedonvälityksen muoto rajapinnoissa verkko-ohjelmointiin liittyen. Se on helppo luettava ihmiselle ja JSON-rakenteen lukemiselle löytyy toteutus monelle eri tekniikalle jo valmiina. Toteutuksessa valinta osoittautui hyväksi ja toimivaksi. Ohjelman tekemä JSON-muoto on nähtävissä liitteessä \ref{ch:report-json-format}.

Osa työtä oli demototeutuksen ongelmien analysointi ja kuinka ne vältettäisiin uudessa versiossa. Ongelmia oli mm. huono suorituskyky, muistivuoto ja toiminnan epävarmuus. Näitä ongelmia analysoitiin työssä syvällisesti kappaleessa \ref{ch:ongelmakohdat-ja-analysointi}. Suorituskykyä saatin toteuksessa paremmaksi valitsemalla suorituskykyisempi C-kieli toteutukseen. C on käännettävä kieli verrattuna Rubyn tulkattavaan kieleen ja C voi hyödyntää käyttöjärjestelmän säikeitä ilman rajoituksia verrattuna Rubyn-tulkin globaaliin lukitukseen (GIL). Muistivuoto saatiin korjattua huolellisella ohjelmoinnilla ja pitämällä kirjaa että muisti oikeasti ohjelmassa vapautetaan kun sitä ei enää tarvita. Toiminnan epävarmuus liittyi yhteyden aikakatkaisuihin ja tämä johtui huonosta suorituskyvystä. Tästä päästiin eroon kun kieli vaihdettiin nopeampaan. Työn toteutuksessa ei ole demossa havaittavia ongelmia ja on osoittanut tuotannossa hyvin toimivaksi muun järjestelmän kanssa.

Tärkeimpänä puutteena toteutuksessa on testiympäristön ja yksikkötestien puuttuminen. Testit olisivat tärkeät ohjelman jatkoa ja ylläpidettävyyttä ajatellen. Etenkin kun tulevaisuudessa siihen tullaan tekemään muutoksia, voidaan yksikkötesteillä varmistaa että ohjelma toimii ainakin niiltä osin halutulla tavalla. Tällä hetkellä tämä täytyisi tehdä käsin testaamalla muutoksien jälkeen. Testejä ei kirjoitettu tämän työn puitteissa, koska aika ei siihen riittänyt. Testit kuitenkin tullaan lisämään ohjelmaan myöhemmin osana muun järjestelmän testiajoa.

Kaikkiaan diplomityössä tuloksena oli järjestelmän erillinen komponentti, joka kykeni tilaamaan viestit yhdeltä IED-laitteelta ja jakamaan tiedon järjestelmän muiden komponenttien kanssa. Kaikki ongelmat mitä demoversiossa oli, saatiin ratkaistua onnistuneesti. Ohjelmisto otettiin käyttöön muun järjestelmän kanssa tuotantoon. Voidaan sanoa, että työ pääsi asetettuihin tavoitteisiin ja onnistui niiltä osin hyvin.