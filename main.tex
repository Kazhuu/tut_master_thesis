% !TeX spellcheck = en_US
\RequirePackage{ifluatex, ifxetex} % these are for the portability of this example - can be omitted in any actual document made for a certain engine

\ifnum 0\ifxetex 1\fi\ifluatex 1\fi>0
\else
  % only needed for using Greek letters outside math when running PDFLaTeX - leave out otherwise
  %\PassOptionsToPackage{LGR}{fontenc}
  %\RequirePackage{textgreek}
\fi


\documentclass[globalnumbering,centeredcaptions,draftfooter]{tutthesis} % see appendix for list of options

%\pagestyle{headings} % Adds titles to the header


% ifnameyear is defined to demonstrate both versions in a single file. You may leave it out and simply use one version throughout your file.
\newif\ifnameyear
\nameyearfalse



% ==============
% Basic packages
% ==============
% You should use these unless you really know what you're doing

\ifnum 0\ifxetex 1\fi\ifluatex 1\fi>0
\else\usepackage[utf8]{inputenc}
\fi

\usepackage[english,finnish]{babel} % The language of the thesis last

% If you are working with a minimal LateX distribution, you may have to install some extra packages. Make sure that at least babel-finnish (available in e.g. texlive-lang-european) and the basic fonts (e.g. texlive-fonts-recommended) are installed.

\usepackage{babelbib} % You should use this unless you are using biblatex. Add option fixlanguage if you're writing in English (the thesis writing guide is asymmetric, requiring Finnish theses to have e.g. 'eds.' for sources in English, while requiring English theses to have all such parts in English)

\ifnameyear\usepackage{natbib} % add option longnamesfirst if you want to have full author list with first citation
\else\providecommand{\citep}{\cite} % This template is written using \citep to get name-year citations right, and in numerical mode the command is here aliased to the standard \cite. If you use numbered citations, leave this out and use \cite
\fi


% ===============
% Useful packages
% ===============
% Packages which are not required for a thesis that follows guidelines, but may be convenient or necessary in common cases

%\usepackage{microtype} % subtle but nice improvements to how text is printed

%\usepackage{textcase} % may be used to keep parts of title lowercase

\usepackage{array}
\usepackage{tabularx} % e.g. multiline cells
%\usepackage{calc} %for performing length arithmetic such as column width = text width minus some other width
%\usepackage{longtable} % for tables spanning multiple pages

%\usepackage{psfrag} % editing ps files
%\usepackage{subfig} % parallel small figures a,b,c,...
%\usepackage{rotating} % for rotating e.g. full-page figures

%\usepackage{siunitx} % nice formatting for combinations of number and unit
\usepackage{amsopn} % For operator names; not necessary if amsmath is used
%\usepackage[fleqn]{amsmath} %Extensions to math handling; if you use this, you should use e.g. gather instead of equation due to a hyperref bug

\usepackage{listings} %Typesetting code
%\lstset{basicstyle=\footnotesize\ttfamily, numbers=left}
%\renewcommand{\lstlistingname}{Ohjelma} % Program if you're writing in English
% If you want non-ASCII characters (e.g. in comments), check out the listingsutf8 package

\ifnum 0\ifxetex 1\fi\ifluatex 1\fi>0
  \usepackage[math-style=ISO]{unicode-math} %must not precede amsmath and most other math and font related packages
\else
  %\usepackage{bm} % The \bm command is used for bold italic variables used in some fields not to be used with unicode-math
  %\usepackage[helvratio=1]{newtxtext} \usepackage{newtxmath}% some recommend the newtx fonts
  \usepackage{textcomp} % symbols like \textdegree
\fi


% ===========================
% Bibliographic information
% ===========================
% These must be set before loading pdfx or beginning document
\author{Mauri Mustonen}
\title{IEC 61850 -standardin MMS raportien tilaus ja jatkoprosessointi}
\datesubmitted{2018}{4}{12} % year, month, day; no leading zeroes; submitted for bachelor's theses and thesisapproved for master’s
\thesistype{Diplomityö} % % Do not use ASCII apostrophe ' as it will not be substituted with the correct one (’) in the PDF metadata. Note that there are both short version (this) and a long one - "Master’s" vs. "Master of Science"
\major{Ohjelmistotuotanto}
\programme{Tietotekniikan koulutusohjelma} % Note apostrophes on all fields for PDF metadata
\examiner{Professori Kari Systä} %\and for plural
%\datetopicapproved{2017}{1}{5} % only for master’s theses
\keywords{Ohjelmistotuotanto, IEC 61850, MMS, AMQP, raportointi}


% Packages that need to be loaded late
% ----------------------------------------------
%\usepackage[a-2u]{pdfx} % If you're using PDFLatex and your version of pdfx is not recent enough, you may run into the inputencoding bug. In that case, load inputenc after pdfx (and replace any non-ASCII characters in the metadata with e.g. \"{a})

\usepackage{hyperref} % This must (usually) be the last package you load - load this OR pdfx (which also loads hyperref). Usage of pdfx would be nice, but if you have issues with that you may fall back to just hyperref



\begin{document}


\maketitle
%First, the abstract in the language of the thesis (no language selection). Note that most fields are already defined.
\thesisdescription{Diplomityö}


\begin{abstract}
Kirjoita yleiskataus tiivistelmä työstä tähän. Kerro lyhyesti mitä työssä tullaan tekemään.
\end{abstract}


%Then, the abstract in the other language (explicit language selection) except for bachelor's theses
\iffalse
\begin{otherlanguage}{english}

\title{Tampere University of Technology thesis template}
\programme{...}
\thesisdescription{...}
\major{Mathematics}
\examiner{Professor Vilma Välkky\and Professor Matti Meikäläinen}
\keywords{thesis, template, thesis structure, thesis layout}

\begin{abstract}
The abstract is a self-contained, concise description of the thesis: what was the problem, what was done, what was the result.
Do not include charts or tables in the abstract.

First include the abstract written in the main language of the thesis and then the translation.
A bachelor's thesis in Finnish must also have a name in English for archival.
\end{abstract}
\end{otherlanguage}
\fi


\chapter*{Alkusanat}

Mistä tämän diplomityönaiheen sain ja kiittää eri ihmisiä ketä työssä oli sidoshenkilöinä.

\vspace{2\baselineskip}

Tampereella, 12.4.2018

\vspace{2\baselineskip}

Mauri Mustonen


% Create table of content.
\tableofcontents


% List of figures and tables.
%\listoffigures
%\listoftables



\chapter*{Lyhenteet ja merkinnät}

% This is not a "proper" table, so no table environment
% Suppressed left colsep; 20% - 1 x colsep; right colpsep; left colpadding; 80% - 1 x colpadding; suppressed right colpadding
\begin{tabular}[h]{@{} p{0.2\textwidth-\tabcolsep} p{0.8\textwidth-\tabcolsep} @{}}
MMS & Manufacturing Message Specification \\
AMQP & Advanced Message Queuing Protocol \\
\end{tabular}


% Each chapter is it's own file and included here.
\chapter{Johdanto}
\label{ch:johdanto}
Kirjoita tähän johdantoa työstä ja aiheesta. Kuinka työ valittiin ja miksi tekijä valitsi tämän työn. Kirjoita myös mitä tehtiin. Kokonaiskuva työstä pitäisi saada johdannosta.


\chapter{MMS-protokolla}
\label{ch:mms-protokolla}
Selitä lyhyesti mikä on MMS-protokolla ja vähän sen tietotyypeistä. Tämän tarkoitus on pohjustaa tulevaa IEC 61850 abstraktien olioiden (ACSI) sovitusta tämän protokollan päälle. Voi pilkkoa tarvittaessa aliotsikoihin.


\chapter{IEC 61850 -standardi}
\label{ch:iec 61850 -standardi}
Kirjoitta yleisesti mikä on IEC 61850 -standardi ja mitä varten se on olemassa. Kerro myös kuinka standardi on pilkottu pienempiin dokumentteihin ja mitä kukin käsittelee.

\section{Standardin abstraktimääritykset}
Kirjoita tähän mitä standardin IEC 61850-7-2 osuudessa määritellään abstraktoimalla fyysisiä laitteita ja palveluita rajapinnoiksi ja olioiksi. Käsittelee standardin Abstract communication service interface (ACSI).

\section{Raportointi}
Kirjoita tähän IEC 61850 standardin määrittästä abstraktista raportointimallista. Tätä raportointi mekanismia tullaan käyttämään raporttien tilauksessa ja täytyy ymmärtää toteuttettavan ohjelmiston kannalta.

\section{Sovitus MMS-protokollaan}
Kirjoita kuinka ylempi ACSI sovitetaan MMS-protokollan palveluiksi ja tietotyypeiksi standardin IEC 61850-8-1 osuuden mukaan. Tähän myös miten raportointi toimii MMS-protokollan päällä.


\chapter{Advanced Message Queuing Protocol}
\label{ch:advanced message queuing protocol}
Kirjoita tähän AMQP määrittävästä standardista, mikä sen tarkoitus on ja mihin sitä voidaan käyttää.

\section{Viestien välitysmekanismit}
Mitä mekanismeja AMQP tarjoaa viestien välittämiseen osapuolille. Näitä on jono, reititys suoraan osapuolien välillä ja viestin julkaisu ja tilaaminen.

\section{Tilaus ja julkaisu -mallin osat}
Kirjoita tähän AMQP tarjoamista viestien julkaisu ja tilaus -mallin osista osapuolten kesken. Kerro mitä eri osat tekevät ja mikä niiden tehtävä viestien välittämisessä on. Englanniksi osia ovat esim. exchange, queue, publisher ja consumer.


\chapter{Lähtökohdat ja tarve}
\label{ch:lähtökohdat ja tarve}
Kirjoita tähän siitä miksi tällainen toteutus tarvitaan ja miksi. Pohjusta miksi suunniteltava ohjelmisto tarvitaan toteuttaa yritykseen johon työn teen. Ohjelmiston tarkoitus olisi tilata IEC 61850 -standardin määrittämiä raportteja ja muokata ne uuteen muotoon (JSON) ja julkaista ne eteenpäin jonoon toiselle tilaavalle ohjelmalle käyttäen AMQP-standardin määrittämää viestintää. Jonon tilaava asiakasohjelmisto voi olla mikä tahansa muu ohjelmisto.

Kirjoita tätä ensin ja listaa eksplisiittisesti mitä ollaan tekemässä ja vaatimuksia. Voi aloitaa heti kirjoittaa ennen kun työn otsikon päättää. Aseta työlle isoja tutkimuskysymyksiä.
\section{Kokonaiskuva}
\section{Kysymykset}
Mikä olisi sopiva arkkitehtuuri tällaisen tilanteen hoitamiseen?

\section{Tilaava asiakasohjelmisto}


\chapter{Toteutettavan arkkitehtuurin suunnittelu}
\label{ch:toteutettavan arkkitehtuurin suunnittelu}
Kirjoittaa tähän kuinka toteutettava arkkitehtuuri suunniteltiin ja kuinka päätöksiin päädytiin. Kirjoitusta myös miten tilattuja raportteja käsitellään ja kuinka niitä julkaistaan eteenpäin. Tarkoituksena olisi saada raportit nykyaikaiseen JSON muotoon.

\section{Raporttien tilaajaohjelmisto}
Kirjoita mitä asiakasohjelman pitää tehdä jotta raportit saadaan tilattua ja mitä parametrejä ohjelmisto tarvitsee toimiakseen. Lisäksi yleisesti mitä ohjelman pitää tehdä ennen raporttien tilausta ja loppullisen halutun muodon saavuttamista.

\section{Uudelleenjulkaistujen raporttien formaatti}
Kirjoita tähän mihin muotoon raportit lopussa tallennetaan esim. JSON. Miksi tähän valintaan päädyttiin. Kerro myös kuinka raportin alkuperäistä rakennetta muokattiin uuteen muotoon sopivaksi.

\section{Raporttien uudelleenjulkaisu}
Kirjoita tähän siitä kuinka AMQP-standardin tilaus ja julkaisu -mallia käytetään raporttien uudelleenjulkaisuun, jotta toinen asiakasohjelma voi niitä ottaa vastaan.


\chapter{Ohjelmiston toteutus}
\label{ch:ohjelmiston toteutus}
Kirjoita tähän osioon siitä kuinka suunniteltu arkkitehtuuri toteutettiin ja millä tekniikoilla. Tämä osio käyttää lyhyitä koodiesimerkkejä hyväkseen selittämään lukijalle kuinka toteutus tehtiin, jotta lukija voisi itse toteuttaa samanlaisen ohjelmiston.

\section{Ohjelmiston toteutuksen valinta}
Kirjoita tähän miksi päädyttiin tietynlaiseen ohjelmiston toteuttamiseen. Työssä on mietitty komentorivipohjaista toteutusta. Lisäksi mille alustalle ohjelmisto suunnitellaan Windows vai Linux.

\section{Kielen valinta}
Kirjoita tähän mikä kieli valittiin toteutuksen tekemiseen ja miksi tämä. Alustava suunnitelma on toteuttaa C-kielellä.

\section{RabbitMQ}
Kirjoita tähän RabbitMQ toteutuksesta. Kirjasto toteuttaa AMQP standardin määrittämiä eri viestintämalleja. Kerro kuinka sitä hyödynnetään tässä työssä ja vähän sen että mitä vaatii.

\section{Käytettävät kirjastot}
Kirjoita tähän erilaisista kirjastoista mitä toteutukseen valittiin ja miksi. Alaotsikoita voi lisätä jos toteutukseen tarvitaan muita kirjastoja.

\subsection{libiec61850}
IEC 61850 -standardin toteuttava C-kirjasto joka tekee raskaan työn standardin määrittämien palveluiden toteuttamiseen ja muodostamiseen. Kirjasto tarjoaa rajapinnat serveri- ja asiakasohjelmiston toteuttamiseen, mutta vain asiakasohjelmiston rajapintoja käytetään. Kirjasto tarjoaa myös rajapinnat haluttujen raporttien tilaamista varten. Kirjaston nettisivu täältä: http://libiec61850.com/libiec61850/.

\subsection{rabbitmq-c}
RabbitMQ:n rajapinnan toteuttava kirjasto C-kielen ohjelmille. Kirjastolla voidaan toteuttaa julkaisevia ja tilaavia ohjelmistoja. Kirjastosta käytetään julkaisevan puolen toteutusta. Kirjasto löytyy täältä: https://github.com/alanxz/rabbitmq-c.

\subsection{JSON-formatointi}
Joku kirjasto JSON formatointiin C-kielelle. Näkyy olevan parikin vaihtoehtoa. Perustele tähän valinta ja miksi.

\section{Jatkokehitys}
Kirjoita tähän ideoita mitä jää jatkokehitykseen ja mitä ohjelmistossa on puutteita tai mitä jäi tekemättä.


\chapter{Tilaava asiakasohjelmisto}
\label{ch:tilaava asiakasohjelmisto}
Kirjoita tähän jonon tilaajan asiakasohjelmasta. Työssä asiakasohjelmistoa ei toteuteta. Vain julkaisijan osuus, mutta voisi esimerkkinä toteuttaa simppelin asiakasohjelman, joka vain tulostaisi saadut viestit. Tai vaihtoehtoisesti kirjoittaa miten asiakasohjelma saa tilattua tietoa pienen esimerkin kautta.


\chapter{Yhteenveto ja tulokset}
\label{ch:yhteenveto ja tulokset}
Kirjoita tähän ensin arviointi ja yhteenveto työstä ja sen lopputuloksista. Mitä hyötyjä työnantaja työstä saa ja jatkokehitysideoita. Mitä työssä meni hyvin ja mitä olisi voinut tehdä toisin. Peilaa myös tuloksia asetettuihin kysymyksiin alussa ja miten niihin päästiin. Sulje työn ympyrä tässä kappaleessa peilaamalla alkutilannetta loppuun.


% This can be deleted later.
\begin{it}
	Kommentteja työtä aloittaessa:
	\begin{itemize}
		\item Olisiko hyvä, että lähdet työssäsi erilaisista hajautus paradigmoista (push vs pull; message queue), perustelet valintasi ja sitten menet suunnitteluun ja toteutukseen?
		\item Ja olisi hyvä, että työ perustelee miksi tuota MQ arkkitehtuuria yleensä (ja rabbitMQ:ta) käytetään.
	\end{itemize}
	
	Things to do now:
	\begin{itemize}
		\item Laittaa aihe hyväksyntään.
		\item Lähde kirjoittamaan teoriaa ja ennen sitä yleistä tasoa missä ollaan. Yleinen korkea taso sen takia, että lukija ymmärtää mistä edes on kyse. Pidä koko ajan kirjoittaessa mielessä top-down lähestymistapa! Erittäin tärkeä!!!
		\item Loppu otsikoida niin että ensin on tulokset, niiden arviointi ja yhteenveto mainitussa järjestyksessä.
		\item Kirjoittaessa miettiä asioita mistä kirjoitetaan ja pitää kontekstista kiinni.
		\item Pidä lauseet simppelineinä ja helppolukuisina! Älä turhaan vaikeuta hommaa lukijalle ja se ei tuo työhön yhtään mitään lisäarvoa! Todella tärkeä asia ajatella! Jos lause käsittää monta asiaa, pilko se pienempiin erillisiin lauseisiin.
		\item Muihinkin lähteisiin voi viitata kuin tieteellisiin. Toki yritä löytää tieteellisiä julkaisuja mahdollisuuksien mukaan. Osoittaa että olet perehtynyt asiaan paremmin.
		\item Kun kirjoitat asiaa esim. että entisessä ohjelmassa oli ongelma että ei skaalaudu helposti tai on huono suorityskyky. Kerro mistä johtopäätös tulee. Tämä ei ole lukijalle selvää tietoa.
		\item Teorien ja yleisen osuuden kirjoittamisen jälkeen, sovi palaveri Karin kanssa.
		\item Työn otsikko on hyvä, ei tarvitse olla erikseen "ohjelmallisesti" sanaa.
		\item Työn päätason otsikoita laittaa enemmän kuvaavimmiksi kuin "Alkutilanne" ja "Teoria".
		\item Käytä työssä viesti sanaa raportin sijaan. Tuo lukijalle esille että se tarkoittaa standardin mukaisia raportteja.
	\end{itemize}
	
	Huomioituja asioita toisten dipoissa:
	\begin{itemize}
		\item Tärkeät sanat esitellään tekstissä ensimmäisen kerran kursiivilla painottamisen takia. Tämän jälkeen ei enää samaa sanaa kursivoida.
		\item Todella paljon erilaisia lähteitä käytetty! Blogiposteja, kirjoja, ja tapahtumien kirjoituksia (IEEE). On myös nettisivuja käytetty lähteenä kun mainitaan esim. Git ja jotain muita sivuja. Nämä tietysti voi olla myös alaliitteenä sivulla.
		\item Tosi hyvin kirjoitettu! Todella selkeää tekstiä ja etenee hyvin ja on lukijalle ystävällinen.
		\item Johdanto on pilkottu otsikoihin työn alkutilanteen selvittämiseksi hyvin ennen teoriaa. Ja teoriaosuus alkaa joustavasti johdannon jälkeen järkevästi.
		\item Kun listataan tekstiä, sana on ensin \emph{kursiivilla} ja on selitetty asiaa. Kohta loppuu puolipisteeseen (;). Tämän jälkeen jatkuu pienellä seuraava aihe ja päättyy myös puolipisteeseen. Viimeinen kohta alkaa myös pienellä, mutta päättyy pisteeseen normaalisti. Seuraava kappale alkaa normaalisti. Listassa lauseet muokkautuvat yhteen esim. käyttäen ja sanaa.
		\item Kysymys teoriassa mikä työssä on jäljellä oli kirjoitettu \emph{kursiivilla}.
	\end{itemize}
\end{it}


% This adds used sources chapter.
\addto\extrasenglish{\btxifchangecaseoff} % Controls the case-changing for English titles. Make sure that case is preserved for abbreviations and proper nouns, e.g. title={The {ABC} of {Tex}: An Introduction to the Typesetting System}

\ifnameyear
  \bibliographystyle{babapaliktutnat}
\else
  \bibliographystyle{bababbrtut}
\fi
\bibliography{references}


% Starts the appendix part.
\appendix

% Appendix chapters here.
%\chapter{Testiliite}
%\label{ch:classdocumentation}


\end{document}