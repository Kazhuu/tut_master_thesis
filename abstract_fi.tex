\begin{abstract}
\begin{it}
	Tiivistelmä on suppea, 1 sivun mittainen itsenäinen esitys työstä: mikä oli ongelma, mitä tehtiin ja mitä saatiin tulokseksi.
	Kuvia, kaavioita ja taulukoita ei käytetä tiivistelmässä.
	
	Laita työn pääkielellä kirjoitettu tiivistelmä ensin ja käännös sen jälkeen.
	Suomenkieliselle kandidaatintyölle pitää olla myös englanninkielinen nimi arkistointia varten.	
\end{it}

Sähkönjakeluverkko on tärkeä osa nykyistä yhteiskuntaa ja sen päivittäistä toimintaa. Siksipä verkon toiminta on elintärkeää yhteiskunnan sujuvan toiminnan kannalta. Sähköverkko koostuu sähköntuotantolaitoksista, sähkölinjoista ja sähköasemista. Sähköverkon eri komponenttien avulla sähkö toimitetaan tuontantolaitoksesta kuluttajan seinäpistokkeeseen asti. Sähköasemat ja niiden automatisointi ovat tärkeässä roolissa verkon yleisen toiminnan ja turvallisuuden takaamiseksi.

Nykypäivänä sähköverkon sähköasemien automatisointiin käytetään tietokoneita, kommunikointi- ja verkkolaitteistoja \cite[s.~659]{Blaine2001}. Asemien automatisointi toteutetaan \emph{älykkäillä elektronisilla laitteilla} (eng. IED), jotka toteuttavat aseman monitoroinnin ja ohjaamisen. IED:t konfiguroidaan toteuttamaan aseman erilaisia funktioita ja toiminnallisuuksia. Monen eri toimijan toimiessa allalla, voisivat kaikki vapaasti toteuttaa oman version laitteistaan ja määrittää kuinka näitä käytettäisiin. Tästä seurauksena olisi, että laitteet eivät olisi yhteensopivia muiden toimittajien laitteiden kanssa. Tilannetta helpottamaan ja mahdollistamaan eri valmistajien yhteensopivuuden laitteiden välillä. On \emph{International Electrotechnical Commission} määrittänyt standardin nimeltä IEC 61850. Standardi määrittää kommunikointi protokollat sähköasemien IED laitteille. Standardi määrittää abstrakteja malleja sähköaseman kommunikoinnin mallintamiseen oliopohjaisesti. Tekniikasta tai toteutuksesta riippumaton standardimalli voidaan määrittää toimivaksi eri tekniikoilla. Standardi määrittää malleja myös erilaisten datapisteiden tilaukseen viesteinä IED:ltä. Eri asiakastoteutukset voivat tilata laitteeseen määritettyjä datapisteitä standardin määrittämillä säännöillä. Standardin ansiosta viestien muoto on ennaltamäärätty ja on tekniikasta riippumaton. Asiakasohjelmisto voi käyttää viestejä esimerkiksi jännitemittatietojen tilaukseen.

Tässä diplomityössä keskitytään yllämainitun viestejä tilaavan asiakasohjelmiston suunnitteluun ja toteutukseen. Työ sisältää tutkimusta ja vertailua arkkitehtuuriin ja toteutukseen liittyen. Lopullinen toteutus asiakasohjelmistosta tilaa viestejä IED:ltä, prosessoi ja uudelleenjulkaisee ne myöhempää käyttöä varten. Ennen työn aloitusta, ohjelmistosta oli jo saatavilla protoversio, jota hyödynnettiin uuden toteutuksen suunnittelussa. Protoversiossa olevat ongelmat ja puutteet korjattiin uuteen toteutukseen.

\end{abstract}