\begin{abstract}
Sähkönjakeluverkko on tärkeä osa nykyistä yhteiskuntaa ja sen päivittäistä toimintaa. Sähköverkko koostuu sähköntuotantolaitoksista, sähkölinjoista ja sähköasemista. Sähköverkon eri komponenttien avulla sähkö toimitetaan tuotantolaitoksista kuluttajille. Sähköasemat ja niiden automatisointi ovat tärkeässä osassa verkon yleisen toiminnan ja turvallisuuden takaamiseksi. Tässä diplomityössä keskitytään suunnittelemaan ja toteuttamaan hajautetun järjestelmän arkkitehtuuri ja ohjelmistokomponentti osaksi isompaa sähköasemiin liittyvää järjestelmää. Toteutuksen tarkoitus on tilata tietoa sähköasemalta verkon yli ja saada jaettua se järjestelmän muiden komponenttien kanssa. Sähköasemalta tuleva tieto on esimerkiksi mittaustietoa, joka näytetään käyttöliittymässä.

Sähköasemilta tieto tilataan älykkäiltä elektroniikkalaitteilta (engl. Intelligent Electronic Device, lyhennetään IED). IED:t ovat sähköaseman verkkoon kytkettyjä automaatiolaitteita, joista käytetään myös nimeä suojarele. IED-laitteiden kommunikointiin liittyy vahvasti maailmanlaajuinen IEC 61850 -standardi (engl. International Electrotechnical Commission). Standardi määrittää kuinka IED-laitteet ja niihin yhteydessä olevien ohjelmien täytyy kommunikoida verkon yli.

Työssä arkkitehtuuri suunnitellaan analysoimalla erilaisia hajautetun järjestelmän kommunikointiparadigmoja ja selvitetään, mitkä sopisivat tarkoitukseen parhaiten. Ennen työn aloitusta ohjelmasta oli toteutettu demo, joka todisti osien toimivuutta. Demototeutuksessa oli ongelmia, jotka haittasivat sen jatkokehitystä. Tässä työssä demoa käytetään pohjana uuden version suunnittelulle. Demosta analysoidaan sen ongelmia ja mistä ne johtuivat. Näitä tietoja käytetään uuden komponentin tekniikan suunnitteluun.

Tuloksena on hajautetun järjestelmän arkkitehtuuri ja ohjelmistokomponentti, joka kykenee tilaamaan viestejä IED-laitteelta IEC 61850 -standardin mukaisesti. Komponentti kykenee prosessoimaan ja jakamaan tilatut viestit järjestelmän muiden komponenttien kanssa.
\end{abstract}