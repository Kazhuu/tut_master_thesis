\begin{abstract}
Sähkönjakeluverkot ovat tärkeä osa nykyistä yhteiskuntaa ja sen päivittäistä toimintaa. Sähköverkko koostuu sähköntuotantolaitoksista, sähkölinjoista ja sähköasemista. Sähköverkon eri komponenttien avulla sähkö toimitetaan tuotantolaitoksista kuluttajille. Sähköasemat ja niiden automatisointi ovat tärkeässä osassa verkon yleisen toiminnan ja turvallisuuden takaamiseksi. Tässä diplomityössä keskitytään suunnittelemaan ja toteuttamaan hajautetun järjestelmän arkkitehtuuria hyödyntävä ohjelmistokomponentti. Komponentin on tarkoitus olla osa isompaa sähköasemiin liittyvää järjestelmää. Sen tilaa tietoa sähköasemalta verkon yli ja jakaa sen järjestelmän muiden komponenttien kanssa. Sähköasemalta tuleva tieto esimerkiksi sisältää mittaustietoa, jota näytetään käyttöliittymässä.

Sähköasemilta tieto tilataan älykkäiltä elektroniikkalaitteilta (engl. Intelligent Electronic Device, IED). IED:t ovat sähköaseman verkkoon kytkettyjä automaatiolaitteita, joista käytetään myös nimitystä suojarele. IED-laitteiden kommunikointiin liittyy maailmanlaajuinen IEC 61850 -standardi (engl. International Electrotechnical Commission). Standardi määrittää kuinka IED-laitteet ja niihin yhteydessä olevat ohjelmat kommunikoivat verkon yli.

Ohjelmasta oli toteutettu demo ennen työn aloitusta, joka todisti osien toimivuuden. Demototeutuksessa oli ongelmia, jotka haittasivat sen jatkokehitystä. Tässä työssä demoa käytetään pohjana uuden version suunnittelulle. Demosta analysoidaan sen ongelmia ja mistä ne johtuivat. Näitä tietoja käytetään uuden komponentin tekniikan suunnitteluun.

Tuloksena on hajautetun järjestelmän arkkitehtuuria hyödyntävä ohjelmistokomponentti, joka kykenee tilaamaan viestejä IED-laitteelta IEC 61850 -standardin mukaisesti. Komponentti kykenee prosessoimaan ja jakamaan tilatut viestit järjestelmän muiden komponenttien kanssa.
\end{abstract}