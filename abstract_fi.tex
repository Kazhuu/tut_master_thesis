\begin{abstract}
% TODO: Muokkaa tätä vastaamaan uutta kirjoitusta mikäli tarve vaatii.
Sähkönjakeluverkko on tärkeä osa nykyistä yhteiskuntaa ja sen päivittäistä toimintaa. Sähköverkko koostuu sähköntuotantolaitoksista, sähkölinjoista ja sähköasemista. Sähköverkon eri komponenttien avulla sähkö toimitetaan tuontantolaitoksesta kuluttajille. Sähköasemat ja niiden automatisointi ovat tärkeässä roolissa verkon yleisen toiminnan ja turvallisuuden takaamiseksi. Tässä diplomityössä keskitytään suunnittelemaan ja toteuttamaan yksittäinen ohjelmistokomponentti osaksi isompaa sähköasemiin liittyvää järjestelmää. Suunniteltavan komponentin tarkoituksena on tilata tietoa sähköasemalta verkon yli ja saada jaettua tämä tieto järjestelmän muille komponenteille. Sähköasemalta tuleva tieto on esimerkiksi mittaustietoa ja mittaustiedosta kiinnostunut järjestelmän komponentti tarvitsee tämän tiedon käyttöliittymässä näyttämiseen.

Sähköasemilta tieto tilataan \emph{älykkäiltä elektroniikkalaitteilta} (engl. \emph{Intelligent Electronic Device}, \emph{IED}). IED:t ovat sähköaseman automaatiolaitteita, jotka on kytketty aseman verkkoon. Näistä käytetään myös nimitystä suojarele. IED-laitteiden kommunikointiin liittyy vahvasti maailmanlaajuinen \emph{IEC 61850} -standardi (engl. \emph{International Electrotechnical Commission}). Standardi määrittää kuinka IED-laitteet kommunikoivat verkon yli ja mekanismit kuinka ulkopuolinen ohjelma voi tilata siltä viestejä.

Ennen työn aloitusta ohjelmasta oli toteutettu demo, joka todisti kokonaisuuden toimivuuden. Demototeutuksessa oli kuitenkin ongelmia, jotka estivät sen käytön luotettavasti tuotannossa. Tässä työssä demoa käytettiin pohjana uuden version suunnittelulle. Demosta analysoitiin sen ongelmia ja mistä ne johtuivat. Näitä tietoja käytettiin uuden komponentin suunnitteluun liittyvissä päätöksissä.

Tuloksena työstä oli muusta järjetelmästä riippumaton ohjelmistokomponentti, joka pystyi tilaamaan viesteja IED-laitteelta IEC 61850 -standardin mukaisesti. Komponentti kykeni prosessoimaan ja jakamaan tilatut viestit järjestelmän muiden komponenttien kanssa. Komponentti päätyi tuotantoon osaksi muuta järjestelmää.
\end{abstract}