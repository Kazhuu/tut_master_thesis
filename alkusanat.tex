\chapter*{Alkusanat}
\label{ch:alkusanat}
Toteutin tämän diplomityöni yritykselle nimeltä Alsus Oy. Alsus oli sen hetkinen työpaikkani vuonna 2018. Diplomintyön aihe liittyi sopivasti sen hetkisiin työtehtäviin ja sisälsi todella paljon oman mielenkiinnon kohteita. Työtehtävistä syntyi idea diplomityöstä, se muokkautui hieman ja lopulta siitä tuli tämän diplomityön aihe. Diplomityöhön liittyvän ohjelmistokehityksen aloitin jo helmikuussa 2018. Ohjelmisto valmistui toukokuussa ja siitä eteenpäin olen käyttänyt aikana työn ohessa diplomityön kirjoittamiseen.

Diplomityössäni aiheena on suunnitella ja kehittää arkkitehtuuria ja ohjelmistokomponentti osaksi isompaa järjestelmää. Järjestelmä liittyi sähköasemiin ja niiden tarkkailuun. Komponentin tarkoituksena on tilata tietoa verkon yli sähköasemilta ja jakaa tieto järjestelmän muiden komponenttien kanssa. Tämä diplomityö on tarkoitettu luettavaksi niille, jotka ovat kiinnostuneita sähköasemien toiminasta tai järjestelmän hajautuksesta ja siihen liittyvistä kommunikointiparadigmoista.

Haluan kiittää Alsus Oy -yritystä aiheesta ja mielenkiintoisista työtehtävistä, jotka mahdollistivat tämän diplomityön. Lisäksi, että sain käyttää työaikaani vapaasti diplomityön tekemiseen ja kirjoittamiseen. Yrityksen puolelta haluan erityisesti kiittää henkilöitä Jouni Renfors ja Samuli Vainio, jotka kannustivat minua ja antoivat palautetta tämän diplomityön tekemiseen. Kiitän työni ohjaajaa professori Kari Systää työni luotettavasta ja hyvästä ohjaamisesta. Haluan myös kiittää läheisiä ystäviäni, joiden kanssa pidimme paljon yhteisiä kirjoitushetkiä ja rakentavia keskusteluja diplomityön tekemisestä. Ilman niitä diplomityöni kirjoitusprosessi olisi venynyt pidemmäksi. Lisäksi kiitän perhettäni saamastani tuesta ja motivaatiosta. Lopuksi kiitän muita tärkeitä ystäviäni, jotka auttoivat minua diplomityössäni oikolukemalla ja motivoimalla minua.

Diplomityöni on kirjoitettu Latex:illa ja kaikki lähdekoodi on saatavissa GitHub:issa osoitteessa: https://github.com/kazooiebombchu/tut\_master\_thesis.

\vspace{2\baselineskip}

Tampereella, 21.11.2018

\begin{figure}[ht!]
	\includegraphics[width=0.3\textwidth,left]{pictures/signature.png}
\end{figure}

Mauri Mustonen