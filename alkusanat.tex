\chapter*{Alkusanat}
\label{ch:alkusanat}
Toteutin tämän diplomityöni yritykselle nimeltä Alsus Oy. Alsus oli sen hetkinen työpaikkani vuonna 2018. Diplomintyön aihe liittyi sopivasti sen hetkisiin työtehtäviin ja sisälsi todella paljon oman mielenkiinnon kohteita. Työtehtävistä syntyi idea diplomityöstä, se muokkautui hieman ja lopulta siitä tuli tämän diplomityön aihe. Diplomityöhön liittyvän ohjelmistokehityksen aloitin jo helmikuussa 2018. Ohjelmisto valmistui toukokuussa ja siitä eteenpäin olen käyttänyt aikani töissä ja tämän työn kirjoittamiseen.

Diplomityössäni aiheena oli suunnitella ja kehittää yksittäinen ohjelmistokomponentti osaksi isompaa järjestelmää. Järjestelmä liittyi sähköasemiin ja niiden tarkkailuun. Komponentin tarkoituksena oli tilata tietoa verkon yli sähköasemilta ja jakaa tieto järjestelmän muiden komponenttien kanssa. Tämä diplomityö on tarkoitettu luettavaksi niille, jotka ovat kiinnostuneita sähköasemiin liittävästä ohjelmistoista tai kuinka isompaa järjestelmää hajautetaan ja tietoa siinä jaetaan.

Haluan kiittää Alsus Oy yritystä aiheesta ja mielenkiintoisista työtehtävistä, jotka mahdollistivat tämän diplomityön. Lisäksi, että sain käyttää työaikaani vapaasti diplomityön tekemiseen ja kirjoittamiseen. Yrityksen puolelta haluan erityisesti kiittää henkilöitä Jouni Renfors ja Samuli Vainio, jotka kannustivat minua ja antoivat palautetta tämän diplomityön tekemiseen. Kiitän työni ohjaajaa professori Kari Systää työni luotettavasta ja todella hyvästä ohjaamisesta. Haluan myös kiittää läheisiä ystäviäni, joiden kanssa pidimme paljon yhteisiä kirjoitushetkiä ja rakentavia keskusteluja diplomityön tekemisestä. Ilman niitä diplomityöni kirjoitusprosessi olisi venynyt pidemmäksi. Lisäksi kiitän perhettäni tuesta ja motivaatiosta koko opiskelujen aikana mitä heiltä sain. Lopuksi kiitän muita tärkeitä ystäviäni, jotka auttoivat minua diplomityössäni oikolukemalla ja motivoimalla minua.

\vspace{2\baselineskip}

Tampereella, 9.8.2018

\vspace{2\baselineskip}

Mauri Mustonen