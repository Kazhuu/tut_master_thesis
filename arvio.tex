\chapter{Tulosten arviointi ja pohdinta}
\begin{it}
Peilaa tuloksia tulevaisuuteen nähden ja mitä se niistä täyttää. Esimerkiksi että tulevaisuuden tavoitteita on otettu huomioon

Kirjoita jatkosta että mahdollisesti voisi omaksi kokonaisuudeksi toteuttaa, mutta pitäisi vielä parantaa kohtia esim. yksi prosessi voisi tilata enemmänkin IED-laitteita. Tämä vaatisi myös että prosessien välinen kommunikointi pitäisi toteuttaa jotenkin toisin kuin parametreilla.
\end{it}

Diplomityössä toteutettu ohjelma ei ole ollut tuotannossa osana muuta järjestelmää vielä kovin kauan. Kuitenkin tähän mennessä se on toiminut ongelmitta. Varmasti ei voida arvioida, että ohjelmassa ei tulisi ongelmia tulevaisuudessa, mutta ainakin alun perusteella tulokset näyttävät lupaavilta. Tältä osin voidaan arvoida, että diplomityö pääsi asetettuihin tavoitteisiin onnistuneesti ja halutut vaatimukset saatiin täytettyä. Kuitenkin toimivuudesta huolimatta toteutuksesa on kohtia mitä voitaisiin parantaa, tehdä toisin ja jatkokehittää. Nämä ovat kuitenkin tulevaisuudessa yrityksen sisällä tehtäviä työtehtäviä tai mahdollisesti toisen diplomityön aiheita.

IEC 61850 -standardin rajoitteiden ja asetettujen vaatimusten pohjalta suunniteltiin hajautetun järjestelmän arkkitehtuuri viestien jakamiseen järjestelmässä. Järjestelmää hajautettiin AMQP-pohjaisella välittäjäpalvelimella ja julkaisija-tilajaa-kommunikointiparadigman avulla. Toteutettu arkkitehtuuri oli esitetty kuvassa \ref{fig:planned-system-architecture}. Tuloksena suunnitellulla arkkitehtuurilla on muitakin hyötyjä kuin työssä käytetyn järjestelmän kanssa. Arkkitehtuuria olisi mahdollista hyödyntää muissakin järjestelmissä, jotka käyttävät IEC 61850 -standardin mukaista kommunikointia ja jakavat tätä tietoa muiden osien kanssa. Periaatteessa suunnitelmasta olisi mahdollista toteuttaa oma kokonaisuutensa, jota olisi mahdollistaa käyttää minkä tahansa järjestelmän kanssa.

Työn aikana järjestelmän hajautukseen pohdittiin eri paradigmojen sopivuutta ja huomattiin, että siihen sopisivat julkaisija-tilaaja-, joukkokommunikointi- ja viestijono-paradigmat. Toteutukseen valittiin AMQP-standardi, joka mahdollisti julkaisija-tilaaja- ja viestijono-paradigmat, mutta ei suoraan ollut tarkoitettu joukkokommunikointiin. Tämän takia joukkokommunikointi jätettiin pois ja korvattiin julkaisija-tilaaja-paradigmalla. Kuinka hyvin joukkokommunikointi olisi sopinut toteutukseen ei ole tarkkaa tietoa. Kuitenkin julkaisija-tilaajan-paradigma jatkaa IEC 61850 -standardin määrittämää julkaisija-tilaaja-kommunikointia IED-laitteen kanssa ja näin ollen sopii toteutukseen hyvin. Toteutukseen myös harkittiin MQTT-standardia AMQP-standardin sijaan, joka on pelkästään julkaisija-tilaaja-kommunikointiin tarkoitettu protokolla. Tämä valinta tehtiin tekijän aikaisemman kokemuksen pohjalta ja muiden yrityksessä olevien henkilöiden keskustelun pohjalta. Koska joukkokommunikointi jätettiin pois toteutuksesta, olisi MQTT voinut sopia toteutukseen paremmin kuin AMQP sen keveyden takia.

IED-laitteelta tuleva viesti päätettiin muuntaa JSON-muotoon XML:län sijaan. Vertailua kahden välillä tehtiin ja päätös oli aikaisemmin tutkimuksen perusteella selvä. JSON-muoto on kevyempi kuin XML ja sopii nykypäivänä viesti muotona hajautettuun järjestelmään todella hyvin. Suunniteltu JSON-rakenne on toiminut käytössä olemisen ajan tarpeiden mukaan. Kuitenkin siinä olisi kohtia mitä pystyisi toteuttamaan toisin. Esimerkiksi bit-string tyypin bittijärjestys (engl endian) voi vaihdella attribuuttien välillä ja tämän takia siitä JSON-viestiin julkaistiin kaksi eri arvoa \emph{valueLittleEndian} ja \emph{valueBigEndian} (liite \ref{ch:report-json-format} rivit 23--24). Käytännössä vastuu muuttujan oikein lukemisesta siirretään tilaajalle. Standardissa kuitenkin on määritetty kuinka päin attribuutti esitetään jos se on tyyppiä bit-string. Tämän vastuun voisi mahdollisesti siirtää rcb\_sub-ohjelman puolelle ja tarjota JSON-viestissä pelkkä \emph{value}-kenttä, niin kuin kaikille muillekin attribuuteille. Tämän lisäksi aikatyyppi \emph{utc-time} JSON-viestissä päätettiin antaa siinä muodossa missä ne tulevat IED-laitteelta, eli millisekunteja UNIX-ajanlaskusta. JSON ei määritä käytettävää aikaformaattia, mutta JSON rajapintoihin suositellaan käytettäväksi ISO 8601 -standardin aikaformaattia \cite{json-api-specification}.

Ennen varsinaista toteutusta demoon liittyviä ongelmien analyysista saatiin hyviä tuloksia suorituskykyyn liittyen. Näiden tietojen pohjalta ohjelman kieleksi valittiin C-kieli suorituskyvyn takia. Ongelmana demon suorituskyvyssä ei pelkästään ollut IED-laitteelta tulevien viestien määrä ja libIEC61850-kirjaston lukitus funktiokutsuissa. Todennkäköisesti suurin syy oli Ruby-oletustulkin GIL, joka mahdollistaa vain yhden säikeen suorituksen kerrallaan ja estää rinnakkaisuuden. C-kielen valinta oli hyvä ratkaisu. Ohjelman aika kaikkien RCB-instanssien tilaamiseen saatiin alas n. 30 sekunnista alle 15 sekuntiin. Suurin osa ajasta tulee IED-laitteille tehtävien kutsujen määrästä. Demon muistinkäyttö Ruby on Rails -ympäristössä oli noin 150 Mt. Rcb\_sub:in muistin käyttö saatiin noin 4 kt, joka on todella iso muutos aikaisempaan nähden. Tekniikan valinnan suhteen päätökset onnistuivat hyvin.

Ohjelman kehityksen aikana pidettiin tietoturvaa mielessä C-ohjelmoissa jolla vältetään sen yleiset virheet, esimerkiksi tekstin formatointihyökkäys \cite{format-string-attack} ja muistin ylivuoto \cite{buffer-overflow-attack}. Kuitekin tietoa järjestelmään tulee ulkopuolelta IED-laitteelta ja voisi mahdollisesti sisältää sille vahingollista tietoa. Tämä osuus ei kuitenkaan kuulunut tämän diplomityön aiheen piiriin. Tietoturvaa toteutuksessa pitäisi kuitenkin tulevaisuudessa käydä enemmän läpi ja tarkemmin.

% Dynaamisesta prosessista ja parametrien antamisesta kertova kappale.

% Tulevaisuudesta kertova kappale ja samalla jatkokehitysideoista omaksi järjestelmäkseen.

Saatuja tuloksia:
\begin{itemize}
	\item näitä olivat komentoriviparametrit, olisiko joku muu tiedonvälitys ollut parempi?
	\item dynaaminen processi olisi ollut parempi kuin yksittäinen rcb\_sub-prosessi, mutta vaikeuttaisi kehitysta rinnakkaisuuden takia
	\item rcb\_sub-ohjelmaa ja RabbitMQ-palvelinta voisi käyttää muussakin järjestelmässä, joka kerää tietoa IED-laitteelta
\end{itemize}

Kysymyksiä:
\begin{itemize}
	\item \emph{Mitkä eri hajautetun järjestelmän kommunikointiparadigmoista sopivat työn vaatimuksien asettaman ongelman ratkaisuun ja mitkä eivät?}
	\item \emph{Minkälainen on hajautetun järjestelmän ohjelmistoarkkitehtuuri joka täyttää asetetut vaatimukset?}
	\item \emph{Järjestelmän hajautuksessa, mikä olisi sopiva tiedon jakamisen muoto eri osapuolten välillä?}
	\item \emph{Mitkä olivat syyt demoversion suorituksen ongelmiin ja kuinka nämä estetään uudessa versiossa?}
\end{itemize}