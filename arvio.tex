\chapter{Tulosten arviointi ja pohdinta}
\begin{it}
Kirjoita tähän yleisesti työn arviosta ja eri tuloksien vertailusta. Tähän voi myös kirjoittaa jatkoheitysideoista mitä voitaisiin vielä parantaa. Tarkoituksena osoittaa lukijalle että on syvällisesti pohdittu eri vaihtoehtoja työn aikana. Jos tuntuu tähän voi myös yhdistää jatkokehityksen toteutuskappaleesta.

Voisiko tähän selittää vaihtoehtoisista toteutustavoista kuten olisiko ollut parempi käyttää konfiguraatiotiedostoja?

Peilaa tuloksia tulevaisuuteen nähden ja mitä se niistä täyttää. Esimerkiksi että tulevaisuuden tavoitteita on otettu huomioon.

Kirjoita tänne myös lopullisen ohjelman käyttämästä muistin määrästä verrattuna rubylla tehtyyn ohjelmaan.
\end{it}
