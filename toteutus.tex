\chapter{Toteutus}
\label{ch:toteutus}
\begin{it}
	Kirjoita tähän osioon siitä kuinka suunniteltu arkkitehtuuri toteutettiin ja millä tekniikoilla. Tämä osio käyttää lyhyitä koodiesimerkkejä hyväkseen selittämään lukijalle kuinka toteutus tehtiin, jotta lukija voisi itse toteuttaa samanlaisen ohjelmiston.
\end{it}

\section{Ohjelmiston toteutuksen valinta}
\begin{it}
	Kirjoita tähän miksi päädyttiin tietynlaiseen ohjelmiston toteuttamiseen. Työssä on mietitty komentorivipohjaista toteutusta. Lisäksi mille alustalle ohjelmisto suunnitellaan Windows vai Linux.
\end{it}

\section{Kielen valinta}
\begin{it}
	Kirjoita tähän mikä kieli valittiin toteutuksen tekemiseen ja miksi tämä. Alustava suunnitelma on toteuttaa C-kielellä.
\end{it}

\section{RabbitMQ}
\begin{it}
	Kirjoita tähän RabbitMQ toteutuksesta. Kirjasto toteuttaa AMQP-standardin määrittämiä eri viestintämalleja. Kerro kuinka sitä hyödynnetään tässä työssä ja vähän sen että mitä vaatii.
\end{it}

\section{Käytettävät kirjastot}
\begin{it}
	Kirjoita tähän erilaisista kirjastoista mitä toteutukseen valittiin ja miksi. Alaotsikoita voi lisätä jos toteutukseen tarvitaan muita kirjastoja.
\end{it}

\subsection{libiec61850}
\begin{it}
	IEC 61850 -standardin toteuttava C-kirjasto joka tekee raskaan työn standardin määrittämien palveluiden toteuttamiseen ja muodostamiseen. Kirjasto tarjoaa rajapinnat serveri- ja asiakasohjelmiston toteuttamiseen, mutta vain asiakasohjelmiston rajapintoja käytetään. Kirjasto tarjoaa myös rajapinnat haluttujen raporttien tilaamista varten. Kirjaston nettisivu täältä: http://libiec61850.com/libiec61850/.
\end{it}

\subsection{rabbitmq-c}
\begin{it}
	RabbitMQ:n rajapinnan toteuttava kirjasto C-kielen ohjelmille. Kirjastolla voidaan toteuttaa julkaisevia ja tilaavia ohjelmistoja. Kirjastosta käytetään julkaisevan puolen toteutusta. Kirjasto löytyy täältä: https://github.com/alanxz/rabbitmq-c.
\end{it}

\subsection{JSON-formatointi}
\begin{it}
	Joku kirjasto JSON formatointiin C-kielelle. Näkyy olevan parikin vaihtoehtoa. Perustele tähän valinta ja miksi.
\end{it}

\section{Jatkokehitys}
\begin{it}
	Kirjoita tähän ideoita mitä jää jatkokehitykseen ja mitä ohjelmistossa on puutteita tai mitä jäi tekemättä.
\end{it}