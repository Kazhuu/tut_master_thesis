\chapter{Alkutilanne}
\label{ch:alkutilanne}
Pohjusta miksi suunniteltava ohjelmisto tarvitaan toteuttaa yritykseen johon työn teen. Alustava suunniteltu ohjelmiston toteutus olisi tilata IEC 61850 -standardin määrittämiä raportteja ja muokata ne uuteen muotoon ja julkaista ne eteenpäin tilaavalle ohjelmalle käyttäen AMQP-standardin määrittämää viestintää. Jonon tilaava asiakasohjelmisto voi olla mikä tahansa muu ohjelmisto. Viestien lopullinen muoto voisi olla JSON.

\section{Kokonaiskuva}
Kirjoita tähän osioon kokonaiskuva alkutilanteesta missä oltiin ennen työn aloittamista. Selvennä kuvilla alkutilanteen arkkitehtuuria.

\section{Ratkaistavat ongelmat}
Kirjoita tähän mitä ongelmia edellisen toteutuksen kanssa on ja mitä yritään ratkoa. Mainitse suorituskykyongelmista Rubylla ja libiec61850-kirjastoa käyttäen.

\section{Tutkimuskysymykset}
Esitä tässä työlle asetettuja tutkimuskysymyksiä. Näitä voisi olla esim. seuraavat:
\begin{itemize}
	\item Mikä on syynä huonoon suorituskykyyn alkutilanteen toteutuksella?
	\item Kuinka suorituskyky paremmaksi verrattuna nykyiseen toteutukseen?
	\item Mitkä ohjelmiston arkkitehtuurin suunnittelumallit (design patterns) olisivat sopivia tämän kaltaisen ongelman ratkaisemiseen? Mitä niistä pitäisi käyttää ja mitä ei?
	\item Mikä olisi sopiva lopullisen prosessoidun tiedon muoto?
	\item Kuinka järjestelmä hajautetaan niin että tiedon siirto eri osapuolten välillä on mahdollista ja joustavaa (push vs pull, message queue jne.)?
\end{itemize}