\begin{abstract}
Electric grids are an important part of the society today. It consists of power plants, power lines and substations. These components allow delivering electricity from power plants to the end users. Substations and their automation play an important role in guaranteeing power grid safety and functionality. The focus of this master thesis is to plan and implement a software component that utilizes a distributed system architecture. The component is intended to be a part of a larger system related to the substations. It subscribes information from the substation and shares it with other parts of the system. The information from the substation can include different types of data, such as measurement data which is shown on the user interface.

The information is subscribed from an Intelligent Electronic Device (IED) in substations. An IED is an automation device which controls the other physical devices of the substation. IEDs are connected to the substation's local network. IED can also be referred to as "protection relay". The International Electrotechnical Commission has defined a worldwide standard called IEC 61850 that defines the rules for how IED-devices and software outside of the substation need to communicate with each other over the network.

A proof of concept software component had already been developed before the start of this thesis. However, this component had many problems which did not encourage its further development. Analyzing these problems is a part of this thesis. The new information is then used to plan the new software's technical side.

The result of this thesis is a software component that utilizes a distributed system architecture. The component can subscribe information from the IED according to the IEC 61850 standard and share it with the other parts of the system.
\end{abstract}