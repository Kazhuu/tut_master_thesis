\begin{abstract}
Nowadays an electric grid is an important part of our society. It consists of power plants, power lines and substations. With these components electricity can be delivered from power plants to the end users. Substations and their automation play an important role in guaranteeing power grid safety and functionality. The focus of this master thesis is to plan and implement distributed system architecture and a software component. Both to be a part of the bigger system which is related to substations and their management. The implementation should be able to subscribe information from the substation and share it with other parts of the system. The information from the substation can include different types of data such as measurement data which for example can be shown on the user interface.

The information is subscribed in substations from an Intelligent Electronic Device, IED for short. An IED is an automation device which controls the other physical devices of the substation. IEDs are also connected to the substation's local network. IED can also be called with the name protection relay. The International Electrotechnical Commission has defined a worldwide standard called IEC 61850 which defines the rules how IED devices should communicate with each other over the substation network. This standard also defines rules for how a software outside the substation network need to communicate with them.

Distributed system architecture is planned by analyzing different communication paradigms and selecting ones that seem to solve the problem best. Before this thesis started, a proof of concept software component had already been developed. However, this component had many problems which did not encourage its development further. Analyzing these problems is a part of this thesis. The new knowledge will help to plan new software's technical side.

As a result from this thesis is a distributed system architecture and a software component. Implementation is able to subscribe information from the IED according to the IEC 61850 standard and share it with the other parts of the system.
\end{abstract}