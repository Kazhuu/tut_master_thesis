\begin{it}
	Kommentteja työtä aloittaessa:
	\begin{itemize}
		\item Olisiko hyvä, että lähdet työssäsi erilaisista hajautus paradigmoista (push vs pull; message queue), perustelet valintasi ja sitten menet suunnitteluun ja toteutukseen?
		\item Ja olisi hyvä, että työ perustelee miksi tuota MQ arkkitehtuuria yleensä (ja rabbitMQ:ta) käytetään.
	\end{itemize}
	
	Things to do now:
	\begin{itemize}
		\item Laittaa aihe hyväksyntään.
		\item Lähde kirjoittamaan teoriaa ja ennen sitä yleistä tasoa missä ollaan. Yleinen korkea taso sen takia, että lukija ymmärtää mistä edes on kyse. Pidä koko ajan kirjoittaessa mielessä top-down lähestymistapa! Erittäin tärkeä!!!
		\item Loppu otsikoida niin että ensin on tulokset, niiden arviointi ja yhteenveto mainitussa järjestyksessä.
		\item Kirjoittaessa miettiä asioita mistä kirjoitetaan ja pitää kontekstista kiinni.
		\item Pidä lauseet simppelineinä ja helppolukuisina! Älä turhaan vaikeuta hommaa lukijalle ja se ei tuo työhön yhtään mitään lisäarvoa! Todella tärkeä asia ajatella! Jos lause käsittää monta asiaa, pilko se pienempiin erillisiin lauseisiin.
		\item Muihinkin lähteisiin voi viitata kuin tieteellisiin. Toki yritä löytää tieteellisiä julkaisuja mahdollisuuksien mukaan. Osoittaa että olet perehtynyt asiaan paremmin.
		\item Kun kirjoitat asiaa esim. että entisessä ohjelmassa oli ongelma että ei skaalaudu helposti tai on huono suorityskyky. Kerro mistä johtopäätös tulee. Tämä ei ole lukijalle selvää tietoa.
		\item Teorien ja yleisen osuuden kirjoittamisen jälkeen, sovi palaveri Karin kanssa.
		\item Työn otsikko on hyvä, ei tarvitse olla erikseen "ohjelmallisesti" sanaa.
		\item Työn päätason otsikoita laittaa enemmän kuvaavimmiksi kuin "Alkutilanne" ja "Teoria".
		\item Käytä työssä viesti sanaa raportin sijaan. Tuo lukijalle esille että se tarkoittaa standardin mukaisia raportteja.
	\end{itemize}
\end{it}