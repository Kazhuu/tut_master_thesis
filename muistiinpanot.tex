Huomioituja asioita toisten dipoissa:
\begin{itemize}
	\item Tärkeät sanat esitellään tekstissä ensimmäisen kerran kursiivilla painottamisen takia. Tämän jälkeen ei enää samaa sanaa kursivoida.
	\item Todella paljon erilaisia lähteitä käytetty! Blogiposteja, kirjoja, ja tapahtumien kirjoituksia (IEEE). On myös nettisivuja käytetty lähteenä kun mainitaan esim. Git ja jotain muita sivuja. Nämä tietysti voi olla myös alaliitteenä sivulla.
	\item Tosi hyvin kirjoitettu! Todella selkeää tekstiä ja etenee hyvin ja on lukijalle ystävällinen.
	\item Johdanto on pilkottu otsikoihin työn alkutilanteen selvittämiseksi hyvin ennen teoriaa. Ja teoriaosuus alkaa joustavasti johdannon jälkeen järkevästi.
	\item Kun listataan tekstiä, sana on ensin \emph{kursiivilla} ja on selitetty asiaa. Kohta loppuu puolipisteeseen (;). Tämän jälkeen jatkuu pienellä seuraava aihe ja päättyy myös puolipisteeseen. Viimeinen kohta alkaa myös pienellä, mutta päättyy pisteeseen normaalisti. Seuraava kappale alkaa normaalisti. Listassa lauseet muokkautuvat yhteen esim. käyttäen ja sanaa.
	\item Kysymys teoriassa mikä työssä on jäljellä oli kirjoitettu \emph{kursiivilla}.
\end{itemize}

Muistiinpanoja:
\begin{itemize}
	\item Avainsanoja käyttää tiivistelmässä.
	\item Kuvat saada sijoitettua oikein missä ne on tekstissä, saada latexin automaattinen sijoittelu pois päältä.
	\item Selvitä onko johdannossa abstraktin tekstiä joka kuuluu sinne kun abstraktia kirjoittaa.
	\item Lauserakenteet paikoin monimutkaisia. Lopussa käy työ läpi ja pilko lauseita.
	\item Otsikointi ei ole tasaista lähdeluettelossa. Tiivistiä tai muokkaa teksitä lopuksi, että rakenne näyttää tasaiselta silmäillä.
	\item Pitääkö johdanto tai muita osia kirjoittaa persoonamuodossa? Tällä hetkellä se on kirjoitettu menneessä ja kolmannessa persoonamuodossa.
\end{itemize}