\begin{it}
	Kommentteja työtä aloittaessa:
	\begin{itemize}
		\item Olisiko hyvä, että lähdet työssäsi erilaisista hajautus paradigmoista (push vs pull; message queue), perustelet valintasi ja sitten menet suunnitteluun ja toteutukseen?
		\item Ja olisi hyvä, että työ perustelee miksi tuota MQ arkkitehtuuria yleensä (ja rabbitMQ:ta) käytetään.
	\end{itemize}
	
	Things to do now:
	\begin{itemize}
		\item Laittaa aihe hyväksyntään.
		\item Lähde kirjoittamaan teoriaa ja ennen sitä yleistä tasoa missä ollaan. Yleinen korkea taso sen takia, että lukija ymmärtää mistä edes on kyse. Pidä koko ajan kirjoittaessa mielessä top-down lähestymistapa! Erittäin tärkeä!!!
		\item Loppu otsikoida niin että ensin on tulokset, niiden arviointi ja yhteenveto mainitussa järjestyksessä.
		\item Kirjoittaessa miettiä asioita mistä kirjoitetaan ja pitää kontekstista kiinni.
		\item Pidä lauseet simppelineinä ja helppolukuisina! Älä turhaan vaikeuta hommaa lukijalle ja se ei tuo työhön yhtään mitään lisäarvoa! Todella tärkeä asia ajatella! Jos lause käsittää monta asiaa, pilko se pienempiin erillisiin lauseisiin.
		\item Muihinkin lähteisiin voi viitata kuin tieteellisiin. Toki yritä löytää tieteellisiä julkaisuja mahdollisuuksien mukaan. Osoittaa että olet perehtynyt asiaan paremmin.
		\item Kun kirjoitat asiaa esim. että entisessä ohjelmassa oli ongelma että ei skaalaudu helposti tai on huono suorityskyky. Kerro mistä johtopäätös tulee. Tämä ei ole lukijalle selvää tietoa.
		\item Teorien ja yleisen osuuden kirjoittamisen jälkeen, sovi palaveri Karin kanssa.
		\item Työn otsikko on hyvä, ei tarvitse olla erikseen "ohjelmallisesti" sanaa.
		\item Työn päätason otsikoita laittaa enemmän kuvaavimmiksi kuin "Alkutilanne" ja "Teoria".
		\item Käytä työssä viesti sanaa raportin sijaan. Tuo lukijalle esille että se tarkoittaa standardin mukaisia raportteja.
	\end{itemize}
	
	Huomioituja asioita toisten dipoissa:
	\begin{itemize}
		\item Tärkeät sanat esitellään tekstissä ensimmäisen kerran kursiivilla painottamisen takia. Tämän jälkeen ei enää samaa sanaa kursivoida.
		\item Todella paljon erilaisia lähteitä käytetty! Blogiposteja, kirjoja, ja tapahtumien kirjoituksia (IEEE). On myös nettisivuja käytetty lähteenä kun mainitaan esim. Git ja jotain muita sivuja. Nämä tietysti voi olla myös alaliitteenä sivulla.
		\item Tosi hyvin kirjoitettu! Todella selkeää tekstiä ja etenee hyvin ja on lukijalle ystävällinen.
		\item Johdanto on pilkottu otsikoihin työn alkutilanteen selvittämiseksi hyvin ennen teoriaa. Ja teoriaosuus alkaa joustavasti johdannon jälkeen järkevästi.
		\item Kun listataan tekstiä, sana on ensin \emph{kursiivilla} ja on selitetty asiaa. Kohta loppuu puolipisteeseen (;). Tämän jälkeen jatkuu pienellä seuraava aihe ja päättyy myös puolipisteeseen. Viimeinen kohta alkaa myös pienellä, mutta päättyy pisteeseen normaalisti. Seuraava kappale alkaa normaalisti. Listassa lauseet muokkautuvat yhteen esim. käyttäen ja sanaa.
		\item Kysymys teoriassa mikä työssä on jäljellä oli kirjoitettu \emph{kursiivilla}.
	\end{itemize}
	
	Palautetta Hannulta:
	Tiivistelmään ottaa mallia tutki ja kirjoita kirjasta. Siinä on hyvin mainittu mitä tiivistelmän pitää sisältää ja pysyä aiheessa. Lainaa tämä kirja kirjastosta.
	Älä jaarittele itsetäänselvyyksiä tiivistelmässä. Esim. kaksi ekaa lausetta on tuhlattu jo saman asian sanomiseen.
	Saako "sähköntuotantolaitoksista, sähkölinjoista ja sähköasemista" yhdistettyä jotenkin, että ei toista samaa sanaa kolme kertaa?
	Avainsanoja käyttää tiivistelmässä.
	Johdannosta pois turha puhuminen itsestäänselvyyksistä. Esim. "Nykypäivänä sähköverkot ovat iso yhteiskuntaa ja sen sujuvaa toimivuutta. Ilman sähköä ei moni asia nykypäivänä toimisi niinkuin se on. Sähköä tarvitaan joka paikassa ja tietotekniikan lisääntyessä vieläkin enemmän." nämä lauseet ovan vähän turhia. Mene suoraan aiheeseen.
\end{it}