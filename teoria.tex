\chapter{Teoria}
\label{ch:teoria}
\begin{it}
	Tähän kohtaan kirjoittaa teoriaa mitä tarvitaan työn toteutuksen ymmärtämisen kannalta. Kaikki työssä tarvittava teoria kuvataan tämän otsikon alla.
\end{it}

\section{IEC 61850 -standardi}
\begin{it}
	Kirjoitta yleisesti mikä on IEC 61850 -standardi ja mitä varten se on olemassa. Kerro myös kuinka standardi on pilkottu pienempiin dokumentteihin ja mitä kukin käsittelee.
\end{it}

\subsection{Standardin määrittämä abstraktimalli}
\begin{it}
	Kirjoita tähän mitä standardin IEC 61850-7-2 osuudessa määritellään abstraktoimalla fyysisiä laitteita ja palveluita rajapinnoiksi ja olioiksi. Käsittelee standardin Abstract communication service interface (ACSI).
\end{it}

\subsection{Viestiblokin konfigurointi ja tilaus}
\begin{it}
	Kirjoita tähän IEC 61850 -standardin määrittästä abstraktista raportointimallista. Tätä raportointi mekanismia tullaan käyttämään raporttien tilauksessa ja sen konfigurointi täytyy ymmärtää toteuttettavan ohjelmiston kannalta.
\end{it}

\subsection{Viestin rakenne}
\begin{it}
	Kirjoita tähän standardin määritämästä viestin rakenteesta ja mitä tietoa se sisältää. Kerro myös sen vaihtoehtoisista kentistä.
\end{it}

\section{Abstraktimallin sovitus MMS-protokollaan}
\begin{it}
	Kirjoita kuinka ylempi ACSI sovitetaan MMS-protokollan palveluiksi ja tietotyypeiksi standardin IEC 61850-8-1 osuuden mukaan. Tähän myös miten raportointi toimii MMS-protokollan päällä.
\end{it}

\subsection{MMS-protokolla}
\begin{it}
	Selitä lyhyesti mikä on MMS-protokolla ja vähän sen tietotyypeistä. Tämän tarkoitus on pohjustaa tulevaa IEC 61850 abstraktien olioiden (ACSI) sovitusta tämän protokollan päälle.
\end{it}

\section{Advanced Message Queuing Protocol}
\begin{it}
	Kirjoita tähän AMQP määrittävästä standardista, mikä sen tarkoitus on ja mihin sitä voidaan käyttää.
\end{it}

\subsection{Viestien välitysmekanismit}
\begin{it}
	Mitä mekanismeja AMQP tarjoaa viestien välittämiseen osapuolille. Näitä on jono, reititys suoraan osapuolien välillä ja viestin julkaisu ja tilaaminen.
\end{it}

\subsection{Tilaus ja julkaisu -mallin osat}
\begin{it}
	Kirjoita tähän AMQP tarjoamista viestien julkaisu ja tilaus -mallin osista osapuolten kesken. Kerro mitä eri osat tekevät ja mikä niiden tehtävä viestien välittämisessä on. Englanniksi osia ovat esim. exchange, queue, publisher ja consumer.
\end{it}