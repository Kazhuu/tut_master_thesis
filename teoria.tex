\chapter{Teoria}
\label{ch:teoria}
\begin{it}
	Tähän kohtaan kirjoittaa teoriaa mitä tarvitaan työn toteutuksen ymmärtämisen kannalta. Kaikki työssä tarvittava teoria kuvataan tämän otsikon alla.

	Tämän lukemalla lukija ymmärtää:
	\begin{itemize}
		\item Mikä on standardin ja sähköaseman relaatio ja kuinka ne liittyvät toisiinsa?
		\item Mikä on IEC standardi ja työn kannalta sen tärkeimmät piirtteet.
		\item Kuinka viestien tilaus tapahtuu standardin mukaan?
		\item Kuinka standardi määrittää viestien rakenteen ja parameterit?
		\item Mikä on MMS-protokolla ja kuinka mallinnus standardin objekteista tehdään sille?
	\end{itemize}
\end{it}

Tässä osiossa lukijaa perehdytetään työn kannalta tärkeään teoriaan. Teoriaosuuden kokonaan lukemalla lukija ymmärtää mitä IEC 61850 -standardi tarkoittaa sähköasemien kannalta ja mitä kaikkea se määrittää. Kuinka standardi määrittää tilattavat viestit ja mitä malleja ja palveluita niihin liittyy. Viestien malli ja toiminta on tärkeä ymmärtää työssä toteutetun ohjelman toiminnan kannalta. Työn lopullisessa toteutuksessa viestit prosessoitiin ja julkaistiin jonopalvelimelle myöhempää käyttöä varten. Teorian lopussa lukija perehdytetään jonopalvelimen toteukseen liityvään teoriaan. Myöskin tämän osuuden teoria on tärkeä ymmärtää ohjelman sisäisen toiminnan kannalta.

\section{IEC 61850 -standardi}
\begin{it}
	Kirjoitta yleisesti mikä on IEC 61850 -standardi ja mitä varten se on olemassa. Kerro myös kuinka standardi on pilkottu pienempiin dokumentteihin ja mitä kukin käsittelee.
\end{it}

Sähköasemilla nykypäivänä käytössä olevilla älykkäillä elektronisilla laitteilla (engl. IED) toteutetaan aseman kontrollointi ja suojaus. Aseman suojauksen lisäksi siihen kuuluu myös asemalta lähtevät sähkölinjat. Jotta suojaus ja ohjaus olisi mahdollista täytyy aseman eri laitteiden kommunikoida verkon yli keskenään ja erilliselle kontrolliasemalle \cite[s.~1]{Brunner2008}. Maailmanlaajuisesti määritetty IEC 61850 -standardi määrittä yhteiset säännöt eri valmistajien laitteiden välille kuinka kommunikointi toteutetaan \cite[s.~10]{IEC61850-7-1}. Ilman yhteisiä sääntöjä jokainen valmistaja olisi vapaa toteuttamaan omat säännöt ja protokollat. Seurauksena olisi, että laitteet eivät olisi keskenään yhteensopivia eri valmistajien välillä. Standardin on tarkoitus poistaa tämä ja määrittää yhteiset pelisäännöt kommunikoinnin toteuttamiseen aseman kommunikointiin liittyen eri laitteiden välillä \cite[s.~1]{Kaneda2008} .

Standardi on määritetty käytettäväksi myös tulevaisuudessa pohjana uusille toteutuksille. Standardi määrittää kaiken abstraktilla tasolla teknisestä toteutuksesta riippumatta. Tämän jälkeen abstraktit mallit ja palvelut voidaan mallintaa eri tekniikoille, joka mahdollistaa sen. \cite[s.~2]{Brunner2008}

\subsection{Standardin historia}
\begin{it}
	Kirjoita tähän standardin historiaa ja kuinka se muodostui. Pohdi kuitenkin ensin onko tämä lukijan kannalta tärkeää ja tarvittavaa tietoa.
\end{it}

\subsection{Standardin eri osat}

IEC 61850 -standardi on todella laaja kokonaisuus. Tämän takia se on pilkottu erillisiin dokumentteihin, josta jokainen käsittelee tiettyä asiaa. Historian saatossa standardiin on lisätty uusia dokumentteja laajentamaan standardia \cite{IEC61850series, New-documents-by-IEC-TC-57}. Tämän työn kirjoitushetkellä standardiin kuului lisäki paljon muitakin dokumentteja, esimerkiksi uusiin mallinnuksiin muille tekniikoille ja vesivoimalaitoksien mallintamiseen liittyviä dokumentteja. Laajuudesta huolimatta standardin voi esittää 10:llä eri pääkohdalla ja näiden alakohdilla. Taulukossa \ref{tab:iec61850-dokumentin-osat} on esitetty standardin pääkohdan dokumentit ja niiden alkuperäiset englanninkieliset otsikot \cite[s.~2]{Mackiewicz2006} \cite{IEC61850series}. Kuvassa \ref{fig:iec61850-osat-ja-relaatiot} on myös esitetty standardin eri osat ja niiden väliset relaatiot.

\begin{table}[ht!]
	\caption{IEC 61850 -standardin pääkohtien ja niiden alakohtien dokumentit.}
	\label{tab:iec61850-dokumentin-osat}
	\begin{tabular}{l | l}
		\hline
		\textbf{Osa} & \textbf{Otsikko englanniksi} \\
		\hline \hline
		1 & Introduction and overview \\
		2 & Glossary \\
		3 & General requirements \\
		4 & System and project management \\
		5 & \parbox[t]{13cm}{Communication requirements for functions and device models} \\
		6 & \parbox[t]{13cm}{Configuration description language for communication in power utility \par automation systems related to IEDs} \\
		7-1 & \parbox[t]{13cm}{Basic communication structure - Principles and models} \\
		7-2 & \parbox[t]{13cm}{Basic information and communication structure - Abstract communication service interface (ACSI)} \\
		7-3 & \parbox[t]{13cm}{Basic communication structure - Common data classes} \\
		7-4 & \parbox[t]{13cm}{Basic communication structure - Compatible logical node classes and data object classes} \\
		8-1 & \parbox[t]{13cm}{Specific communication service mapping (SCSM) - \par  Mappings to MMS (ISO 9506-1 and ISO 9506-2) and to ISO/IEC 8802-3} \\
		9-2 & \parbox[t]{13cm}{Specific communication service mapping (SCSM) - \par  Sampled values over ISO/IEC 8802-3} \\
		9-3 & \parbox[t]{13cm}{Precision time protocol profile for power utility automation} \\
		10 & Conformance testing \\
		\hline
	\end{tabular}
\end{table}

Standardin ensimmäiset osat 1--5 kattavat yleistä kuvaa standardista ja sen vaatimuksista. Osiossa 6 käsitellään IED-laitteiden konfigurointiin käytetty XML (engl. Extensible Markup Language) -pohjainen kieli \cite[s.~7--8]{IEC61850-6}. Tämä osuus ei ole tämän työn kannalta tärkeä ja sitä ei sen tarkemmin käsitellä. Osat 7.1--7.4 käsittelevät standardin abstraktia mallia ja kuinka se rakentuu. Tämä lyhennetään standardissa ACSI (engl. Abstract Communication Service Interface), ja samaa lyhennettä käytetään tässä työssä. Osissa 8--9 ja niiden alakohdissa käsitellään abstraktimallien mallintamista erillisille protokollille, jolloin malleista tulee kyseisestä tekniikasta riippuvaisia. Abstrakteja malleja ja niiden mallintamista tekniikalle käsitellään teoriassa tuonnempana. Osa 10 käsittelee testausmenetelmiä, joilla voidaan varmistaa standardin määritysten noudattaminen. Tämä osuus ei myöskään ole tämän työn kannalta tärkeä ja sitä ei teoriassa sen takia käsitellä.

\begin{figure}
	\includegraphics[width=1\textwidth]{pictures/iec-61850-series-and-relations.png}
	\caption{IEC 61850 -standardin osat ja niiden väliset relaatiot \cite[s.~14]{IEC61850-7-1}.}
	\label{fig:iec61850-osat-ja-relaatiot}
\end{figure}

\subsection{Standardin määrittämä abstraktimalli}
\begin{it}
	Kirjoita tähän mitä standardin IEC 61850-7-2 osuudessa määritellään abstraktoimalla fyysisiä laitteita ja palveluita rajapinnoiksi ja olioiksi. Käsittelee standardin Abstract communication service interface (ACSI).
\end{it}

IEC 61850 -standardin määrittämät konseptit ja abstraktimallit voidaan käyttää myös mallintamaan:
\begin{itemize}
	\item vesivoimalaitos,
	\item sähköasemien välistä kommunikointia,
	\item informaation välitys hajautetun automaation välillä,
	\item informaation välitys sähköaseman ja ohjauskeskuksen välillä,
	\item informaation välitys mittaukseen,
	\item tilan monitorointia ja diagnosointia, ja
	\item informaation välitys konfigurointiin tarkoitetulle tekniikalle \cite[s.~11]{IEC61850-7-1}.
\end{itemize}

\subsection{Viestiblokin konfigurointi ja tilaus}
\begin{it}
	Kirjoita tähän IEC 61850 -standardin määrittästä abstraktista raportointimallista. Tätä raportointi mekanismia tullaan käyttämään raporttien tilauksessa ja sen konfigurointi täytyy ymmärtää toteuttettavan ohjelmiston kannalta.
\end{it}

\subsection{Viestin rakenne}
\begin{it}
	Kirjoita tähän standardin määritämästä viestin rakenteesta ja mitä tietoa se sisältää. Kerro myös sen vaihtoehtoisista kentistä.
\end{it}

\section{Abstraktimallin sovitus MMS-protokollaan}
\begin{it}
	Kirjoita kuinka ylempi ACSI sovitetaan MMS-protokollan palveluiksi ja tietotyypeiksi standardin IEC 61850-8-1 osuuden mukaan. Tähän myös miten raportointi toimii MMS-protokollan päällä.
\end{it}

\subsection{MMS-protokolla}
\begin{it}
	Selitä lyhyesti mikä on MMS-protokolla ja vähän sen tietotyypeistä. Tämän tarkoitus on pohjustaa tulevaa IEC 61850 abstraktien olioiden (ACSI) sovitusta tämän protokollan päälle.
\end{it}

\section{Advanced Message Queuing Protocol}
\begin{it}
	Kirjoita tähän AMQP määrittävästä standardista, mikä sen tarkoitus on ja mihin sitä voidaan käyttää.
\end{it}

\subsection{Viestien välitysmekanismit}
\begin{it}
	Mitä mekanismeja AMQP tarjoaa viestien välittämiseen osapuolille. Näitä on jono, reititys suoraan osapuolien välillä ja viestin julkaisu ja tilaaminen.
\end{it}

\subsection{Tilaus ja julkaisu -mallin osat}
\begin{it}
	Kirjoita tähän AMQP tarjoamista viestien julkaisu ja tilaus -mallin osista osapuolten kesken. Kerro mitä eri osat tekevät ja mikä niiden tehtävä viestien välittämisessä on. Englanniksi osia ovat esim. exchange, queue, publisher ja consumer.
\end{it}