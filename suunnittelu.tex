\chapter{Suunnittelu}
\label{ch:suunnittelu}
Kirjoittaa tähän kuinka toteutettava arkkitehtuuri suunniteltiin ja kuinka päätöksiin päädytiin. Kirjoitusta myös miten tilattuja raportteja käsitellään ja kuinka niitä julkaistaan eteenpäin. Tarkoituksena olisi saada raportit nykyaikaiseen JSON muotoon.

\section{Suorituskyvyn parantaminen}
Miksi entisen toteutuksen suorituskyky on huono ja mitä voitaisiin tehdä sen parantamisesksi. Kirjoita vaikutuksista tähän ja mihin tuloksiin päädyttiin.

\section{Järjestelmän hajautus}
Lähde erilaisista hajautukista (pull vs pull, message queue) ja päätä mikä sopii tähän toteutukseen parhaiten ja miksi. 

\section{Ohjelmiston parametrisointi}
Kirjoita mitä asiakasohjelman pitää tehdä jotta raportit saadaan tilattua ja mitä parametrejä ohjelmisto tarvitsee toimiakseen. Kuinka käyttäjä kontrolloi ohjelman eri asetuksia?

\section{Arkkitehtuurin suunnittelu}
Määritä ohjelman tarkempaa arkkitehtuuria mitä voidaan käyttää asetettujen ja yllämainittujen asioiden saavuttamiseen ja tarkentamiseen. Mitä jos käyttäjä tilaa monta viestiblokkia, niin missä järjestykssä asiat tehdään jne.

\section{Prosessoidun viestin muoto}
Kirjoita tähän mihin muotoon viestit lopussa tallennetaan esim. JSON. Miksi tähän valintaan päädyttiin. Kerro myös kuinka raportin alkuperäistä rakennetta muokattiin uuteen muotoon sopivaksi.