\chapter{Johdanto}
\label{ch:johdanto}
Sähköverkko koostuu tuotantolaitoksista, sähkölinjoista ja sähköasemista. Sähköasemilla on erilaisia tehtäviä verkossa. Näitä ovat esimerkiksi jännitteen muuntaminen, verkon jakaminen ja sen toiminnan tarkkailu. Nykypäivänä asemien toiminnallisuutta voidaan seurata ja ohjata etäohjauksella verkon yli. Tätä kautta etäohjelman on mahdollista tilata tietoa aseman toiminnasta ja sen tilasta. Sähköaseman yksi tärkeä tehtävä on suojata ja tarkkailla verkon toimivuutta, ja esimerkiksi vikatilanteessa katkaista linjasta virrat pois. Tällainen vikatilanne on esimerkiksi kaapelin poikki meneminen, joka aiheuttaa vaarallisen oikosulkutilanteen.

Tässä diplomityössä on tarkoituksena suunnitella hajautetun järjestelmän arkkitehtuuri ja toteuttaa ohjelmisto osaksi isompaa sähköasemiin liittyvää järjestelmää. Tavoitteena on tilata viestejä verkon yli sähköaseman automaatiolaitteelta ja jakaa saadut viestit järjestelmän muiden osien kanssa. Työssä käsitellään hajautetun järjestelmän paradigmoja ja analysoidaan mitkä niistä sopisivat tilanteeseen parhaiten. Analyysin tuloksien ja ohjelmistolle asetettujen vaatimusten perusteella päädytään kokonaisuuden suunnitelmaan. Suunnitelma toteutetaan ohjelmistoksi, joka toimii osana olemassa olevaa isompaa järjestelmää. Toteutus jakaa sähköaseman viestin järjestelmän muille osille, joita ovat esimerkiksi mittaustiedon näyttäminen ja aseman tilan tarkkailu käyttöliittymäkomponenteissa.

Viestit sähköasemilta tilataan sen verkkoon kytketyiltä automaatiolaitteilta. Automaatiolaitteet ohjaavat aseman muuta laitteistoa kuten muuntajia ja katkaisijoita. Nämä automaatiolaitteet noudattavat verkon yli kommunikoinnissa maailmanlaajuisesti määritettyä \emph{IEC 61850 -standardia} (\emph{International Electrotechnical Commission}). Standardin tarkoituksena on määrittää yhteinen kommunikointiprotokolla ja säännöt aseman kaikkien eri laitteiden välille. Standardi määrittää myös asemalta viestien tilaamisen mekanismit, joita aseman ulkopuolisen ohjelman täytyy noudattaa. Nämä määritykset ovat tämän työn kannalta tärkein osa standardia ja vaikuttavat hajautetun järjestelmän suunnitteluun.

Diplomityön tekijä oli jo ennen tämän työn aloitusta Alsus Oy:ssä toteuttanut yksinkertaisen demo-ohjelmiston (proof of concept). Ohjelmisto kykeni tilaamaan viestejä asemalta standardin mukaisesti ja tallentamaan ne tietokantaan. Toteutus oli puutteellinen ja siinä oli toimintaan liittyviä ongelmia, jotka haittasivat sen jatkokehitystä tuotantoon asti. Demon tarkoituksena oli opettaa tekijälle standardia ja sen mekanismeja ennen oikeaa toteutusta. Tässä työssä analysoidaan demon toimintaa ja sen ongelmia. Nämä tulokset yhdistetään aikaisemmin mainitun hajautetun järjestelmän suunnittelun kanssa. Lopputuloksena saadaan toimiva kokonaisuuden suunnitelma uudelle toteutukselle.

Tämän diplomityön rakenne alkaa pohjatietojen käsittelyllä. Ensin käydään läpi järjestelmän hajautuksen teoriaa ja siihen liittyviä kommunikointiparadigmoja. Tämän jälkeen käsitellään IEC 61850 -standardia ja sen toimintaa. Edellä mainittujen tietojen avulla analysoidaan erilaisia hajautuksen vaihtoehtoja ja mitkä niistä sopisivat toteutukseen parhaiten. Tuloksena tästä on hajautetun järjestelmän arkkitehtuurisuunnitelma, jota tekninen suunnittelu tulee tarkentamaan. Ennen teknistä suunnittelua työssä analysoidaan demon toimintaa ja sen ongelmia. Kaikkien edellä mainittujen perusteella suunnitellaan ohjelmistoarkkitehtuuri ja valitaan sen käyttämät tekniikat. Suunnitelman ja päätöksien jälkeen työssä käydään läpi itse ohjelman toteutus. Työn lopussa arvioidaan ja pohditaan saatuja tuloksia asetettuihin tutkimuskysymyksiin ja miten tavoitteisiin päästiin. Lisäksi käsitellään myös toteutuksen tulevaisuutta ja mahdollisia vaihtoehtoisia toteutustapoja.


\section{Vaatimukset}
\label{ch:vaatimukset}
Diplomityön suunnittelulle ja toteutukselle asetettiin vaatimuksia, jotka ohjelmiston pitää pystyä täyttämään. Vaatimuksien tehtävä on asettaa työlle selvät tavoitteet mitä järjestelmään halutaan tuoda lisää. Diplomityön tehtävä on tiedon analyysin ja tutkimustyön kautta löytää sopivat menetelmät suunnitteluun ja sen toteutukseen. Asetettuja vaatimuksia käytetään pohjana työssä tehdyille päätöksille. Pohdintaosiossa vaatimuksilla arvioidaan työn tavoitteisiin pääsyä ja eri ratkaisuvaihtoehtoja. Kaikki vaatimukset oli asetettu jo ennen työn aloittamista tai sen alkuvaiheessa. Jokaiselle vaatimukselle on asetettu lyhenne, jolla siihen viitataan muualla tekstissä.

Ohjelmistolle asetettiin mm. seuraavia vaatimuksia:
\begin{itemize}
	\item V1: viestien tilaus sähköasemalta IEC 61850 -standardin mukaisesti,
	\item V2: tilattujen viestien jakaminen järjestelmässä siitä kiinnostuvien komponenttien kanssa,
	\item V3: tilattuja viestejä haluavien komponenttien määrä pitää pystyä muuttumaan järjestelmän tarpeiden mukaan,
	\item V4: muu järjestelmä ohjaa milloin viestien tilaus sähköasemilta aloitetaan ja lopetetaan,
	\item V5: komponenttien täytyy saada ilmoitus uudesta viestistä ilman erillistä kyselyä,
	\item V6: viestit puskuroidaan myöhempää käsittelyä varten jos komponentti ei ehdi niitä heti käsitellä,
	\item V7: komponentin pitää pystyä suodattamaan viestit sen lähteen identiteetin perusteella,
	\item V8: viestien jakamisen muoto pitää olla helposti ymmärrettävä järjestelmän osapuolien kesken,
	\item V9: sähköasemalta tilattavien viestien määrä pitää pystyä muuttumaan tilauksien välillä,
	\item V10: viestien välitystekniikka järjestelmässä täytyy tukea verkkopalvelun tapauksessa TCP/IP-pro\-to \-kol\-la\-mää\-ri\-tyk\-si\-ä, ja
	\item V11: tiedonsiirrossa lähetystakuu ei ole välttämättömyys.
\end{itemize}

Osa vaatimuksista tulevat muun järjestelmän toimintaperiaatteista ja siitä kuinka sitä käytetään. Käyttäjän on tarkoitus pystyä ohjaamaan viestien tilausta sähköasemille käyttöliittymän kautta. Tästä seurauksena on vaatimus, jossa muun järjestelmän täytyy ohjata viestin hakuun liittyviä osia. Uusien järjestelmän komponenttien kehitys halutaan pitää helppona. Tämän takia viestin saamisen rajapinnat tulee suunnitella komponentissa helposti käytettäväksi. Rajapinnan tulee tarjota viestit helposti ymmärrettävässä muodossa. Lisäksi ilmoitukset uuden viestin saapuessa ja niiden puskurointi, jos niitä ei ehditä heti käsitellä. Komponentille pitää tarjota mahdollisuus erottaa tai saada viestit sähköaseman ja sen automaatiolaitteen mukaan. Tämä sen takia, että järjestelmä käsittelee paljon erilaisia sähköasemia ja niiden automaatiolaitteita. Komponenttien pitää pystyä selvittämään viestien alkuperä. Tästä esimerkkinä on tietyn automaatiolaitteen mittaustiedon näyttäminen. Tiedonsiirtoon ei alustavasti tarvittu lähetystakuita tiedon tyypin takia (mittausdata). Uusi viesti korvaa edellisen lähetyksen tiedot. Olisi tavoiteltavaa, että toteutus kuitenkin mahdollistaisi tämän tulevaisuuden varalta. Muu järjestelmä ja sen osat ovat toteutettu web-sovelluksena. Tästä tulee vaatimus, että viesti täytyy onnistua kuljettamaan osapuolten välillä TCP/IP-protokollaperheen avulla.


\section{Tutkimuskysymykset}
Ohjelmiston vaatimuksien lisäksi diplomityölle asetettiin selviä tutkimuskysymyksiä, joihin työn aikana yritetään etsiä vastausta. Tutkimuskysymykset liittyvät työhön korkealla tasolla ja käsittelevät sen kokonaisuudesta eri kohtia. Tutkimuskysymyksiä käytetään myös työn lopussa yhteenvedossa.

Työlle asetettiin seuraavat tutkimuskysymykset:
\begin{itemize}
	\item T1: \emph{Mitkä eri hajautetun järjestelmän kommunikointiparadigmoista sopivat työn vaatimuksien asettaman ongelman ratkaisuun ja mitkä eivät?}
	\item T2: \emph{Minkälainen on hajautetun järjestelmän ohjelmistoarkkitehtuuri, joka täyttää asetetut vaatimukset?}
	\item T3: \emph{Järjestelmän hajautuksessa, mikä olisi sopiva tiedon jakamisen muoto eri osapuolten välillä?}
	\item T4: \emph{Mitkä olivat syyt demoversion ongelmiin ja kuinka nämä estetään uudessa versiossa?}
\end{itemize}