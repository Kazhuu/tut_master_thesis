\chapter{Johdanto}
\label{ch:johdanto}
Tämä diplomityö oli tehty Alsus Oy:lle, joka oli työn tekohetkellä tekijän työpaikka vuonna 2018. Tekijä valitsi työn aiheen mielenkiinnon ja ajankohdan sopiivuden takia. Työ liittyi sopivasti ajanhetkellä sen hetkisiin työtehtäviin.

Sähköverkko koostuu tuotantolaitoksista, sähkölinjoista ja sähköasemista. Sähköasemien tehtävä verkossa on toteuttaa erilaisia toiminnallisuuksia, kuten muuntaja, jakaminen ja verkon toiminnan tarkkailu. Lisäksi nykypäivänä asemien toiminnallisuutta voidaan seurata ja ohjata etänä. Sähköaseman yksi tärkeä tehtävä on suojata ja tarkkailla verkon toimivuutta ja vikatilanteessa esimerkiksi katkaista linjasta virrat pois. Tällainen vikatilanne voisi olla kaapelin poikkimeno ja virta pääsisi tätä kautta maihin.

Sähköasemien funktionaalisuutta ja ohjausta nykypäivänä toteuttaa niin sanottu älykäs elektroniikkalaite (engl. Intelligent Electronic Device, lyhennetään IED). IED-laite voidaan kytkeä ja konfiguroida toteuttamaan monta aseman eri funktionaalisuutta ja ne on myös kytketty aseman verkkoon. IED:t voivat kommunikoida verkon yli aseman muun logiikan ja muiden IED-laitteiden kanssa, ja näin toteuttaa aseman toiminnallisuutta. Nykypäivänä verkon nopeus ja mahdollistaa reaaliaikaisen kommunikoinnin asemilla sen eri laitteiden välillä. IED-laitteet voivat myös kommunikoida aseman verkosta ulospäin, esimerkiksi keskitettyyn ohjauskeskukseen. Yksi IED-laite voidaan esimerkiksi konfiguroida hoitamaan sähkölinjan kytkimenä oloa, joka myös tarkkailee linjan toimintaa mittaamalla konfiguroituja arvoja, kuten jännitettä ja virtaa. Vikatilanteen sattuessa IED katkaisee linjan virrasta enemmän vahingon välttämiseksi. Linjan korjauksen jälkeen virta kytketään takaisin päälle.

Monen eri toimijan toimiessa laitteita tuottavalla allalla ja sähköasema suuren elektronisen laitteiden määrän takia. On määritetty maailmanlaajuinen standardi nimeltä IEC 61850, jonka tarkoitus on määrittää yhteinen kommunikointiprotokolla aseman kaikkien eri laitteiden ja valmistajien välille. Standardi määrittää eri valmistajien IED:laitteille samat yhteiset kommunikointiprotokollat, joita noudattamalla eri valmistajien laitteet sopivat yhteen.

Standardissa on määritetty säännöt, millä IED-laitteen ulkopuolinen ohjelma voi tilata viestejä verkon yli IED-laitteelta. Tilatut viestit voivat esimerkiksi sisältää mitattuja kolmivaihe jännitteitä tai muuta haluttua tietoa IED-laitteesta ja sen tilasta. Tässä työssä keskityttiin tämän asiakasohjelman suunnitteluun ja toteutukseen. Asiakasohjelman tarkoitus oli tilata viestit, prosessoida ne ja julkaista eteenpäin jonopalvelimelle muille ohjelmille saataviksi. Koska ohjelman toteutukseen tärkeäksi osaksi liittyy IEC 61850, ja käytetyn jonopalvelimen standardit. Käsitellään nämä osiot työn teoriaosuudessa ensin ennen suunnittelua ja toteutusta.

Tämän työn tekijä oli jo ennen tämän työn aloitusta Alsus Oy:ssä toteuttanut yksinkertaiseen demoversion kyseisestä ohjelmasta. Toteutus oli puutteellinen ja siinä oli toimintahäiriöitä, mutta kuitenkin todisti eri osien toimivuuden mahdollisuuden ja opetti tekijää asian suhteen. Tässä työssä tätä demoversiota käytettiin pohjana kokonaan uuden toteutuksen suunnittelulle. Työssä analysoidaan sen toimintahäiriöitä ja mitkä ne aiheutti. Näitä tietoja käytettiin pohjana uuden toteutuksen liittyvien päätöksien tekemiseen.

Tämän työn tutkimustyön osuus on miettiä ja tutkia uuden toteutuksen arkkitehtuuria ja toteutusta. Tarkoitus on täyttää kaikki uudelle toteutukselle asetetut tarpeet ja estää demoversioon liittyvät toimintahäiriöt. Työlle asetettiin tutkimuskysymyksiä, joita peilataan työn lopussa saavutettuihin tuloksiin ja pohditaan kuinka hyvin niihin päästiin. Työlle asetettiin seuraavat tutkimuskysymykset:
\begin{itemize}
	\item Mitkä ohjelmiston arkkitehtuurin suunnittelumallit (engl. design patterns) olisivat sopivia tämän kaltaisen ongelman ratkaisemiseen? Mitä niistä pitäisi käyttää ja mitä ei?
	\item Kuinka järjestelmä hajautetaan niin että tiedon siirto eri osapuolten välillä on mahdollista ja joustavaa (push vs pull, message queue jne.)?
	\item Mitkä olivat syyt demoversion toimintahäiriöihin ja kuinka nämä estetään uudessa toteutuksessa?
	\item Järjestelmän hajautuksessa, mikä olisi sopiva tiedon jakamisen muoto eri osapuolten välillä?
\end{itemize}