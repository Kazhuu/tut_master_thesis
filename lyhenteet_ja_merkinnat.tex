\chapter*{Lyhenteet ja merkinnät}
\label{ch:lyhenteetjamerkinnat}
% This is not a "proper" table, so no table environment
% Suppressed left colsep; 20% - 1 x colsep; right colpsep; left colpadding; 80% - 1 x colpadding; suppressed right colpadding

% Lyhenteiden läpikäyminen jäi sivulle 9
\begin{tabular}[h]{@{} p{0.2\textwidth-\tabcolsep} p{0.8\textwidth-\tabcolsep} @{}}
	ACSI & engl. \emph{Abstract Communication Service Interface}, IEC 61850 -standardin käyttämä lyhenne kuvaamaan palveluiden abstraktimalleja \\
	AMQP & engl. \emph{Advanced Message Queuing Protocol} \\
	CDC & engl. \emph{Common Data Class} on nimitys IEC 61850 -standadissa joukko luokkia joista muodostuu dataobjektien istanssit \\
	DA & engl. \emph{Data Attribute} on IEC 61850 -standardissa käsite abstrahoimaan jokin sähköaseman laitteen mitattava arvo (esim. jännite) \\
	DO & engl. \emph{Data Object} on IEC 61850 -standardissa käsite abstrahoimaan joukko samaan kuuluvia data-attribuutteja \\
	DPC & engl. \emph{Controllable Double Point} on luokkatyyppi IEC 61850 -standardissa nimeltään Pos \\
	FFI & engl. \emph{Foreign Function Interface}, mekanismi, jolla ajettava ohjelma voi kutsua toisella kielellä toteutettua funktiota\\
	GIL & engl. \emph{Global Interpreter Lock}, tulkattavassa kielissä oleva globaali lukitus, joka rajoittaa yhden säikeen suoritukseen kerrallaan \\
	HAL & engl. \emph{Hardware Abstraction Layer}, laitteistoabstraktiotaso abstraktoimaan laitteen toiminnallisuuden lähdekoodista \\
	IEC & engl. \emph{International Electrotechnical Commission}, on sähköalan kansainvälinen standardiorganisaatio \\
	IED & engl. \emph{Intelligent Electronic Device}, sähköaseman älykäs elektroninen laite (myös nimellä turvarele), joka tarjoaa toimintoja monitorointiin ja kontrollointiin \\
	IEC 61850 & maailmanlaajuinen sähköasemien IED-laitteiden kommunikoinnin määrittävä standardi \\
	IP & engl. \emph{Internet Protocol} on protokolla verkkoliikenteessä joka huolehtii pakettien perille toimittamisesta \\
	LD & engl. \emph{Logical Device} on IEC 61850 -standadissa käsite abstrahoimaan joukko fyysisestä laitteesta joukko loogisesti yhteen kuuluvia laitteita \\
	LN & engl. \emph{Logical Node} on IEC 61850 -standadissa käsite abstrahoimaan fyysinen laite loogisen laitteen ryhmästä \\
	MMS & engl. \emph{Manufacturing Message Specification} on maailmanlaajuinen standardi reaaliaikaiseen kommunikointiin verkon yli laitteiden välillä \\
	MV & engl. \emph{Measured Value} on IEC 61850 -standardissa luokkatyyppi dataobjektille \\
	PD & engl. \emph{Physical Device} on IEC 61850 -standardissa käytetty käsite abstrahoimaan sähköaseman fyysinen laite \\
	RCB & engl. \emph{Report Control Block}, raporttien konfigurointiin ja tilaukseen tarkoitettu luokkatyyppi IED-laitteelle \\
	SCSM & engl. \emph{Specific Communication Service Mapping} on IEC 61850 -standardin abstahoitujen mallien toteuttaminen jollakin tekniikalla \\
	TCP/IP & engl. \emph{Transmission Control Protocol/Internet Protocol}, on joukko standardeja verkkoliikenteen määrityksiin \\
	XCBR & on IEC 61850 -standardissa luokka mallintamaan sähkölinjan katkaisijaa (engl. circuit braker) \\
	XML & engl. \emph{Extensible Markup Language}, laajennettava merkintäkieli, joka on ihmis- ja koneluettava \\
	MMXU & on IEC 61850 -standardissa luokka mallintamaan mitattuja arvoja (engl. measurement) \\
\end{tabular}