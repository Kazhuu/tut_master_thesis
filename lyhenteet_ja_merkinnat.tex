\chapter*{Lyhenteet ja merkinnät}
\label{ch:lyhenteetjamerkinnat}

% This is not a "proper" table, so no table environment
% Suppressed left colsep; 20% - 1 x colsep; right colpsep; left colpadding; 80% - 1 x colpadding; suppressed right colpadding
\begin{tabular}[h]{@{} p{0.2\textwidth-\tabcolsep} p{0.8\textwidth-\tabcolsep} @{}}
	ACSI & engl. \emph{Abstract Communication Service Interface}, IEC 61850 -standardin käyttämä lyhenne kuvaamaan palveluiden abstraktimalleja \\
	AMQP & engl. \emph{Advanced Message Queuing Protocol} \\
	FFI & engl. \emph{Foreign Function Interface}, mekanismi, jolla ajettava ohjelma voi kutsua toisella kielellä implementoitua funktiota\\
	GIL & engl. \emph{Global Interpreter Lock}, tulkattavassa kielissä oleva globaali lukitus, joka rajoittaa yhden säikeen suoritukseen kerrallaan \\
	HAL & engl. \emph{Hardware Abstraction Layer}, laitteistoabstraktiotaso abstraktoimaan laitteen toiminnalisuus lähdekoodista \\
	IED & engl. \emph{Intelligent Electronic Device}, sähköaseman älykäs elektroninen laite, joka tarjoaa toimintoja monitorointiin ja kontrollointiin \\
	MMS & engl. \emph{Manufacturing Message Specification} \\
	RCB & engl. \emph{Report Control Block}, raporttien konfigurointiin ja tilaukseen tarkoitettu lohko asiakasohjelmalle \\
\end{tabular}