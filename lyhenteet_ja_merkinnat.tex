\chapter*{Lyhenteet ja merkinnät}
\label{ch:lyhenteetjamerkinnat}
% This is not a "proper" table, so no table environment
% Suppressed left colsep; 20% - 1 x colsep; right colpsep; left colpadding; 80% - 1 x colpadding; suppressed right colpadding
\begin{tabularx}{\linewidth}[h]{@{} p{0.2\textwidth-\tabcolsep} p{0.8\textwidth-\tabcolsep} @{}}
	ACSI & engl. \emph{Abstract Communication Service Interface}, IEC 61850 -standardin käyttämä lyhenne kuvaamaan palveluiden abstraktimalleja \\
	AMQP & engl. \emph{Advanced Message Queuing Protocol} on avoin standardi viestien välitykseen eri osapuolien kesken \\
	BRCB & engl. \emph{Buffered Report Control Block} on IEC 61850 -standardissa puskuroitu viestien tilaamisesta vastaava luokka \\
	CDC & engl. \emph{Common Data Class} on IEC 61850 -standardissa joukko uudelleenkäytettäviä dataobjekin luokkia \\
	CMV & engl. \emph{Complex Measured Value} on IEC 61850 -standardissa dataobjektin luokkatyyppi \\
	DA & engl. \emph{Data Attribute} on IEC 61850 -standardissa käsite abstrahoimaan jokin sähköaseman laitteen mitattava arvo (esim. jännite) \\
	dchg & engl. \emph{data change} on IEC 61850 -standardissa oleva liipaisimen tyyppi \\
	DO & engl. \emph{Data Object} on IEC 61850 -standardissa käsite abstrahoimaan joukko samaan kuuluvia data-attribuutteja \\
	DPC & engl. \emph{Controllable Double Point} on IEC 61850 -standardissa dataobjektin luokkatyyppi nimeltään Pos \\
	DSM & engl. \emph{Distributed Shared Memory} on jaettu muisti, joka käyttäjälle näyttää kuin paikallinen fyysinen muisti \\
	dupd & engl. \emph{data update} on IEC 61850 -standardissa oleva liipaisimen tyyppi \\
	FC & engl. \emph{Functional Constraint} on IEC 61850 -standardissa käsite viitattujen data-attribuuttien rajoittamiseen \\
	FCD & engl. \emph{Functional Constrained Data} on IEC 61850 -standardissa viitteen tyyppi rajoittaamaan viitattuja data-attribuutteja hierarkiassa ensimmäisestä dataobjektista alaspäin \\
	FCDA & engl. \emph{Functional Constrained Data Attribute} on IEC 61850 -standardissa viitteen tyyppi rajoittamaan data-attribuutteja hierarkiassa muusta kuin ensimmäisestä dataobjektista alaspäin \\
	FFI & engl. \emph{Foreign Function Interface}, mekanismi, jolla ohjelma voi kutsua toisella kielellä toteutettuja funktiota \\
	GCS & engl. \emph{Group Communication System} tarkoittaa systeemiä, jossa kommunikoidaan joukolle osapuolia \\
	GI & engl. \emph{General Interrogation} on IEC 61850 -standardissa oleva liipaisimen tyyppi \\
	GIL & engl. \emph{Global Interpreter Lock}, Ruby-kielen tulkissa oleva globaali tulkkilukitus, joka rajoittaa yhden säikeen suoritukseen kerrallaan \\
	GVL & engl. \emph{Global Virtual Machine Lock} on sama kuin GIL, mutta eri nimellä \\
	HAL & engl. \emph{Hardware Abstraction Layer} on laitteistoabstraktiotaso abstraktoimaan laitteen toiminnallisuuden lähdekoodista \\
	IEC & engl. \emph{International Electrotechnical Commission}, on sähköalan kansainvälinen standardiorganisaatio \\
	IEC 61850 & maailmanlaajuinen sähköasemien IED-laitteiden kommunikoinnin määrittävä standardi \\
	IED & engl. \emph{Intelligent Electronic Device}, sähköaseman älykäs elektroniikkalaite (myös nimellä turvarele), joka toteuttaa aseman automaatiota \\
	IoT & engl. \emph{Internet of Things}, on verkko joka koostu siihen kytketyistä erilaisista laitteista \\
	IP & engl. \emph{Internet Protocol} on protokolla verkkoliikenteessä joka huolehtii pakettien perille toimittamisesta \\
	JRuby & on Ruby-kielen tulkki Ruby-koodin suoritukseen Java-virtuaalikoneella \\
	JSON & engl. \emph{JavaScript Object Notation} on JavaScript-kielessä käytetty notaatio objektista ja sen sisällöstä \\
	JVM & engl. \emph{Java Virtual Machine} on Java-kielen virtuaalikone Java-koodin suoritukseen \\
	LD & engl. \emph{Logical Device} on IEC 61850 -standardissa käsite abstrahoimaan joukko fyysisestä laitteesta joukko loogisesti yhteen kuuluvia laitteita \\
	LN & engl. \emph{Logical Node} on IEC 61850 -standardissa käsite abstrahoimaan fyysinen laite loogisen laitteen ryhmästä \\
	mag & on dataobjektin instanssin data-attribuutti nimeltään mag (engl. magnitude) \\
	MMS & engl. \emph{Manufacturing Message Specification} on maailmanlaajuinen standardi reaaliaikaiseen kommunikointiin verkon yli eri laitteiden välillä \\
	MMXU & engl. \emph{measurement} on IEC 61850 -standardissa loogisen noodin luokka mallintamaan mitattuja arvoja \\
	MQTT & engl. \emph{Message Queuing Telemetry Transport} on julkaisija-tilaaja-pohjainen avoin standardi kommunikointiin hajautetussa järjestelmässä \\
	MRI & engl. \emph{Matz’s Ruby Interpreter} on Ruby-kielen tulkki \\
	MV & engl. \emph{Measured Value} on IEC 61850 -standardissa on dataobjektin luokkatyyppi \\
	OptFlds & engl. \emph{Optional Fields} on attribuutti viestin vaihtoehtoisten kenttien määritykseen \\
	PD & engl. \emph{Physical Device} on IEC 61850 -standardissa käytetty käsite abstrahoimaan sähköaseman fyysinen laite \\
	phsA & dataobjektin instanssi nimeltään phsA (engl. phase A) ja tyyppiä CMV \\
	PhV & dataobjektin instanssi nimeltään phV (engl. phase to ground voltage) ja tyyppiä WYE \\
	Pos & dataobjektin instanssi tyyppiä DPC ja nimeltä Pos (engl. position) \\
	q & dataobjektin instanssin data-attribuutti nimeltään q (engl. quality) \\
	qchg & engl. \emph{quality change} on IEC 61850 -standardissa oleva liipaisimen tyyppi \\
	RCB & engl. \emph{Report Control Block}, raporttien konfigurointiin ja tilaukseen tarkoitettu luokkatyyppi IED-laitteelle \\
	RMI & engl. \emph{Remote Method Invocation} on oliopohjainen metodikutsu jossa metodi sijaitsee toisella koneella \\
	RoR & engl. \emph{Ruby on Rails} on kehys web-sovellusten kehittämiseen Ruby-kielellä \\
	RPC & engl. \emph{Remote Procedure Call} on etäproseduurikutsu jossa proseduuri sijaitsee toisella koneella \\
	SCSM & engl. \emph{Specific Communication Service Mapping} on IEC 61850 -standardin abstrahoitujen mallien toteuttaminen jollakin tekniikalla \\
	stVal & dataobjektin instanssin data-attribuutti nimeltään stVal (engl. status value) \\
	t & dataobjektin instanssin data-attribuutti nimeltään t (engl. timestamp) \\
	TCP/IP & engl. \emph{Transmission Control Protocol/Internet Protocol}, on joukko standardeja verkkoliikenteen määrityksiin \\
	TotW & dataobjektin instanssi tyyppiltään MV ja nimeltä TotW (engl. total active power) \\
	TrgOp & engl. \emph{Trigger Options} on IEC 61850 -standardissa käytetty lyhenne määritetyille liipaisimille \\
	UDP & engl. \emph{User Datagram Protocol} on pakettien lähettämisen protokolla Internet protokollan päällä \\
	URCB & engl. \emph{Unbuffered Report Control Block} on IEC 61850 -standardissa ei puskuroitu viestien tilaamisesta vastaava luokka \\
	WYE & engl. \emph{Phase to ground/neutral related measured values of a three-phase system} on IEC 61850 -standardissa dataobjektin luokkatyyppi \\
	XCBR & on IEC 61850 -standardissa luokka mallintamaan sähkölinjan katkaisijaa (engl. circuit breaker) \\
	XML & engl. \emph{Extensible Markup Language} on laajennettava merkintäkieli, joka on ihmis- ja koneluettava \\
	YARV & engl. \emph{Yet another Ruby VM} on Ruby-kielen toinen tulkki, jonka tarkoitus on korvata MRI-tulkki \\
\end{tabularx}