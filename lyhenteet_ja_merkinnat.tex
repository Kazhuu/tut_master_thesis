\chapter*{Lyhenteet ja merkinnät}
\label{ch:lyhenteetjamerkinnat}
% This is not a "proper" table, so no table environment
% Suppressed left colsep; 20% - 1 x colsep; right colpsep; left colpadding; 80% - 1 x colpadding; suppressed right colpadding
\begin{tabularx}{\linewidth}[h]{@{} p{0.2\textwidth-\tabcolsep} p{0.8\textwidth-\tabcolsep} @{}}
	AMQP & \emph{Advanced Message Queuing Protocol} on avoin standardi viestien välitykseen eri osapuolien kesken \\
	DA & \emph{Data Attribute} on IEC 61850 -standardissa käsite abstrahoimaan jokin sähköaseman laitteen mitattava arvo (esim. jännite) \\
	DO & \emph{Data Object} on IEC 61850 -standardissa käsite abstrahoimaan joukko samaan kuuluvia muuttujia \\
	DSM & \emph{Distributed Shared Memory} on jaettu muisti, joka käyttäjälle näyttää kuin paikallinen fyysinen muisti \\
	FFI & \emph{Foreign Function Interface}, mekanismi, jolla ohjelma voi kutsua toisella kielellä toteutettuja funktiota \\
	GCS & \emph{Group Communication System} tarkoittaa systeemiä, jossa kommunikoidaan joukolle osapuolia \\
	GIL & \emph{Global Interpreter Lock}, Ruby-kielen tulkissa oleva globaali tulkkilukitus, joka rajoittaa yhden säikeen suoritukseen kerrallaan \\
	GVL & \emph{Global Virtual Machine Lock} on sama kuin GIL, mutta eri nimellä \\
	HAL & \emph{Hardware Abstraction Layer} on laitteistoabstraktiotaso abstrahoimaan laitteen toiminnallisuuden lähdekoodista \\
	IEC & \emph{International Electrotechnical Commission}, on sähköalan kansainvälinen standardiorganisaatio \\
	IEC 61850 & maailmanlaajuinen sähköasemien IED-laitteiden kommunikoinnin määrittävä standardi \\
	IED & \emph{Intelligent Electronic Device}, sähköaseman älykäs elektroniikkalaite (myös nimellä turvarele), joka toteuttaa aseman automaatiota \\
	IoT & \emph{Internet of Things}, on verkko joka koostu siihen kytketyistä erilaisista laitteista \\
	IP & \emph{Internet Protocol} on protokolla verkkoliikenteessä joka huolehtii pakettien perille toimittamisesta \\
	JRuby & on Ruby-kielen tulkki Ruby-koodin suoritukseen Java-virtuaalikoneella \\
	JSON & \emph{JavaScript Object Notation} on JavaScript-kielessä käytetty notaatio objektista ja sen sisällöstä \\
	JVM & \emph{Java Virtual Machine} on Java-kielen virtuaalikone Java-koodin suoritukseen \\
	LD & \emph{Logical Device} on IEC 61850 -standardissa käsite abstrahoimaan joukko fyysisestä laitteesta joukko loogisesti yhteen kuuluvia laitteita \\
	LN & \emph{Logical Node} on IEC 61850 -standardissa käsite abstrahoimaan fyysinen laite loogisen laitteen ryhmästä \\
	MMS & \emph{Manufacturing Message Specification} on maailmanlaajuinen standardi reaaliaikaiseen kommunikointiin verkon yli eri laitteiden välillä \\
	MQTT & \emph{Message Queuing Telemetry Transport} on julkaisija-tilaaja-pohjainen avoin standardi kommunikointiin hajautetussa järjestelmässä \\
	MV & \emph{Measured Value} on IEC 61850 -standardissa on dataobjektin luokkatyyppi \\
	RCB & \emph{Report Control Block}, viestien konfigurointiin ja tilaukseen tarkoitettu luokkatyyppi IED-laitteelle \\
	RMI & \emph{Remote Method Invocation} on oliopohjainen metodikutsu jossa metodi sijaitsee toisella koneella \\
	RoR & \emph{Ruby on Rails} on kehys web-sovellusten kehittämiseen Ruby-kielellä \\
	RPC & \emph{Remote Procedure Call} on etäproseduurikutsu jossa proseduuri sijaitsee toisella koneella \\
	TCP/IP & \emph{Transmission Control Protocol/Internet Protocol}, on joukko standardeja verkkoliikenteen määrityksiin \\
	UDP & \emph{User Datagram Protocol} on pakettien lähettämisen protokolla Internet protokollan päällä \\
	XML & \emph{Extensible Markup Language} on laajennettava merkintäkieli, joka on ihmis- ja koneluettava \\
	YARV & \emph{Yet another Ruby VM} on Ruby-kielen toinen tulkki, jonka tarkoitus on korvata MRI-tulkki \\
\end{tabularx}